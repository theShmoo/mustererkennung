%
% Einführung in die Mustererkennung - WS2013
% Abgabeprotokoll Exercise 1
%
%

%{{{ misc
\documentclass[subfigure,epsfig,fleqn,float,ausarbeitung]{scrartcl}

\usepackage{graphicx}
\usepackage{epstopdf}
\usepackage{caption}
\usepackage{subcaption}
\usepackage{amsmath}

\usepackage{pgfplots}

%Zitieren:
\usepackage[english]{babel}
%\usepackage[german]{babel}
\usepackage{babelbib} % für das Erstellen des Bibtex-Literaturverzeichnisses
\usepackage{cite}
%\selectbiblanguage{english}
%\selectbiblanguage{german}

%Peseudocode
\usepackage{algpseudocode}
\usepackage{algorithm}

%Fancy shit
\usepackage{url}

\usepackage[]{mcode}

\usepackage[pdftitle={Einfuehrung in die Mustererkennung, Exercise 3},
 						pdfauthor={Matthias Gusenbauer},
						pdfauthor={Matthias Vigele},
						pdfauthor={David Pfahler},
            pdfsubject={Mustererkennung},
            pdfborder={0 0 0}]{hyperref}


\usetikzlibrary{plotmarks}
\pgfplotsset{compat=newest} 
\pgfplotsset{plot coordinates/math parser=false}

%%%%%%%%%%%%%%%%%%%%%%%%%%%%%%%%
% Titlepage

\pagestyle{empty}


%set dimensions of columns, gap between columns, and paragraph indent

\setlength{\textheight}{24.7 cm}
\setlength{\columnsep}{1 cm}
\setlength{\textwidth}{16 cm}
%\setlength{\footheight}{0.0 cm}
\setlength{\topmargin}{0.0 cm}
\setlength{\headheight}{0.0 cm}
\setlength{\headsep}{-0.3 cm}
\setlength{\oddsidemargin}{0.0 cm}
\setlength{\parindent}{0 cm}
\setlength{\parskip}{0.5em}
\setlength{\mathindent}{0mm}

% set page counter if document is part of proceedings
\setcounter{page}{1}
\renewcommand{\floatpagefraction}{0.9}
\renewcommand{\textfraction}{0.1}

%\renewcommand{\captionlabelfont}{\fontfamily{phv}\fontseries{bx}\fontsize{10}{10pt}\selectfont}
%\renewcommand{\captionfont}{\fontfamily{phv}\fontsize{10}{12pt}\selectfont}
%\setlength{\captionmargin}{0.5 cm}

\makeatletter
\makeatother
\def\RR{\hbox{I\kern-.2em\hbox{R}}}


\begin{document}

%don't want date printed
\date{\today}

%make title bold and 14 pt font (Latex default is non-bold, 16pt) 
\title{~\\
  ~\\
  \fontsize{14}{14pt} \bf Abgabedokument Exercise 3
	 ~\\
  \fontsize{12}{12pt} \bf Einfuehrung in die Mustererkennung 186.840 WS 2013}

%for single author 
\author{~\\
  ~\\
  \fontsize{12}{12pt}
  {\bf David Pfahler, Matthias Gusenbauer, Matthias Vigele}\\
  1126287, 1125577, 1126171
  ~\\ ~\\ ~\\
  \normalsize
}

\maketitle
%I don't know why I have to reset thispagestyle, but otherwise get page numbers 
\normalfont
\thispagestyle{empty}

%%%%%%%%%%%%%%%%%%%%%%%%%%%%%%%%%%%%%%%%%%%%%%%%%%%%%%%%%%%%%%%%%%%%%%%%%%%%%%%%
% CONTENT

\stepcounter{part}
\part{The second Report}
\label{cha:secondReport}

\section{Practical Application}

\subsection{Feature Selection}

\subsection{Training- and Test-Set}

\subsection{Classification}

\subsubsection{Perceptron}

\subsubsection{Mahalanobis Distance Classifier}

\subsubsection{k-NN Classifier}

\subsection{Results}

\subsection{Discussion of the Results}


\section{Neural Network}

The following section describes the our experiences results from the neural network toolbox provided by MATLAB.

\subsection{Toolbox}

MATLAB offers a very intuitive way to use the neural network toolbox. You can either chose to use a graphical user interface where it is very easy to play around with different parameters. Mathworks even provided various predefined datasets ranging from medical data to wine and even the well known iris flower dataset is provided. ~\\
The big advantage of the toolbox is not only the fact that you can solve the problem with the given UI you can change parameters, train, retrain, validate the network, create plots of the results and most important of all you there are several tools provided to dynamically create a script which does exactly what the GUI does. The benefit here is that you can seize complete control over you 

\subsection{Results}

\subsection{Discussion of the Results}


\end{document}

% vim:foldmethod=marker
