%
% Einführung in die Mustererkennung - WS2013
% Abgabeprotokoll Exercise 1
%
%

%{{{ misc
\documentclass[subfigure,epsfig,fleqn,float,ausarbeitung]{scrartcl}

\usepackage{graphicx}
\usepackage{epstopdf}
\usepackage{caption}
\usepackage{subcaption}
\usepackage{amsmath}

\usepackage{pgfplots}

%Zitieren:
\usepackage[english]{babel}
%\usepackage[german]{babel}
\usepackage{babelbib} % für das Erstellen des Bibtex-Literaturverzeichnisses
\usepackage{cite}
%\selectbiblanguage{english}
%\selectbiblanguage{german}

%Peseudocode
\usepackage{algpseudocode}
\usepackage{algorithm}

%Fancy shit
\usepackage{url}

\usepackage[]{mcode}

\usepackage[pdftitle={Einfuehrung in die Mustererkennung, Exercise 3},
 						pdfauthor={Matthias Gusenbauer},
						pdfauthor={Matthias Vigele},
						pdfauthor={David Pfahler},
            pdfsubject={Mustererkennung},
            pdfborder={0 0 0}]{hyperref}


\usetikzlibrary{plotmarks}
\pgfplotsset{compat=newest} 
\pgfplotsset{plot coordinates/math parser=false}

%%%%%%%%%%%%%%%%%%%%%%%%%%%%%%%%
% Titlepage

\pagestyle{empty}


%set dimensions of columns, gap between columns, and paragraph indent

\setlength{\textheight}{24.7 cm}
\setlength{\columnsep}{1 cm}
\setlength{\textwidth}{16 cm}
%\setlength{\footheight}{0.0 cm}
\setlength{\topmargin}{0.0 cm}
\setlength{\headheight}{0.0 cm}
\setlength{\headsep}{-0.3 cm}
\setlength{\oddsidemargin}{0.0 cm}
\setlength{\parindent}{0 cm}
\setlength{\parskip}{0.5em}
\setlength{\mathindent}{0mm}

% set page counter if document is part of proceedings
\setcounter{page}{1}
\renewcommand{\floatpagefraction}{0.9}
\renewcommand{\textfraction}{0.1}

%\renewcommand{\captionlabelfont}{\fontfamily{phv}\fontseries{bx}\fontsize{10}{10pt}\selectfont}
%\renewcommand{\captionfont}{\fontfamily{phv}\fontsize{10}{12pt}\selectfont}
%\setlength{\captionmargin}{0.5 cm}

\makeatletter
\makeatother
\def\RR{\hbox{I\kern-.2em\hbox{R}}}


\begin{document}

%don't want date printed
\date{\today}

%make title bold and 14 pt font (Latex default is non-bold, 16pt) 
\title{~\\
  ~\\
  \fontsize{14}{14pt} \bf Abgabedokument Exercise 3
	 ~\\
  \fontsize{12}{12pt} \bf Einfuehrung in die Mustererkennung 186.840 WS 2013}

%for single author 
\author{~\\
  ~\\
  \fontsize{12}{12pt}
  {\bf David Pfahler, Matthias Gusenbauer, Matthias Vigele}\\
  1126287, 1125577, 1126171
  ~\\ ~\\ ~\\
  \normalsize
}

\maketitle
%I don't know why I have to reset thispagestyle, but otherwise get page numbers 
\normalfont
\thispagestyle{empty}

%%%%%%%%%%%%%%%%%%%%%%%%%%%%%%%%%%%%%%%%%%%%%%%%%%%%%%%%%%%%%%%%%%%%%%%%%%%%%%%%
% CONTENT

\part{The first Report}
\label{cha:firstReport}

\section{Perceptron}
\label{sec:perceptron}

The following section describes how we implemented the online perceptron training algorithm to get the linear discriminant function and evaluates it.

\subsection{Application}
\label{sec:1app}

The \texttt{matlab}-function \texttt{[ w ] = perco( X,t,maxEpoches )} calculates the perceptron weight vectors with the online perceptron training algorithm. The Listing~\ref{lst:perc} shows this algorithm.


\begin{algorithmic}[1]
\State Initialise $w,\gamma$
\Repeat
	\For {$i = 1 : N$}
		\If{$w^T(x_iy_i)\leq 0$(misclassified $i$th pattern)}
			\State $w\gets w + \gamma x_iy_i$
		\EndIf
	\EndFor
\Until{all patterns correctly classified}
\end{algorithmic}
\label{lst:perc}

The \texttt{matlab}-function \texttt{plot\_perco\_results(w,X,classLabel,titleName)} plots the linear discriminant function and the data from the training set labeled by the consigned classes.

Instead of letting the algorithm run until all patterns are correctly classified (see Listing~\ref{lst:perc}) it stops after a given number of iterations. Because this number is not fixed by the exercise description, we used different values and compared the results in the next section.

First we calculate the lineal discriminant function for the data provided by the Exercise (\texttt{perceptrondata.dat}) with the two different target data sets and plot the results (See Figure~\ref{fig:decisionBoundary1} and Figure~\ref{fig:decisionBoundary2} for 10 epochs and Figure~\ref{fig:decisionBoundary1100} and Figure~\ref{fig:decisionBoundary2100} for 100 epochs)

The calculation of the perceptron weight vector (see Equation~\ref{eq:w1}) for the first target data set terminated after 9 epoches, because the whole training data set was classified correctly.

\begin{equation}
w_1 = 
	\begin{bmatrix}
      -4.0000\\
			6.2628\\
			3.3863\\
	\end{bmatrix}
	\label{eq:w1}
\end{equation}

The calculation of the perceptron weight vector (see Equation~\ref{eq:w2}) for the second target data set did not terminate after 100 epoches.

\begin{equation}
w_2 = 
	\begin{bmatrix}
      -2.0000\\
			2.9205\\
			1.5864\\
	\end{bmatrix}
	\label{eq:w2}
\end{equation}

The linear function for the separation of the two classes for the first target data is calculated in Equation~\ref{eq:fSpecial}or for a more general case see Equation~\ref{eq:fGeneral}

\begin{equation}
	f(x,y) = w_1 +w_2 x+w_3 y 
	\label{eq:fSpecial}
\end{equation}
\begin{equation}
	f(\vec{w},\vec{y}) = \vec{w} \cdot \vec{y}
	\label{eq:fGeneral}
\end{equation}

\begin{figure}
	\centering
	\newlength\figureheight 
	\newlength\figurewidth 
	\setlength\figureheight{7cm} 
	\setlength\figurewidth{9cm}
	% This file was created by matlab2tikz v0.4.4 running on MATLAB 8.0.
% Copyright (c) 2008--2013, Nico Schlömer <nico.schloemer@gmail.com>
% All rights reserved.
% 
% The latest updates can be retrieved from
%   http://www.mathworks.com/matlabcentral/fileexchange/22022-matlab2tikz
% where you can also make suggestions and rate matlab2tikz.
% 
%
% defining custom colors
\definecolor{mycolor1}{rgb}{1,0,1}%
%
\begin{tikzpicture}

\begin{axis}[%
width=\figurewidth,
height=\figureheight,
colormap/jet,
unbounded coords=jump,
scale only axis,
xmin=-0.0880227,
xmax=1.099484837,
xlabel={${\text{x}}$},
ymin=-0.0988798014,
ymax=1.090901487,
ylabel={${\text{y}}$},
legend style={draw=black,fill=white,legend cell align=left}
]

\addplot[area legend,solid,draw=mycolor1]
table[row sep=crcr]{
x y\\
-2.75866956626581 6.28318530717959 \\
-2.75047067663685 6.26802202012841 \\
-2.73138153522019 6.23271795531469 \\
-2.70409350417457 6.18225060344979 \\
-2.70000332477195 6.17468609567137 \\
-2.67680547312895 6.1317832515849 \\
-2.64953597290705 6.08135017121433 \\
-2.64951744208333 6.08131589972 \\
-2.62222941103771 6.03084854785511 \\
-2.59906862104216 5.9880142467573 \\
-2.59494137999209 5.98038119599021 \\
-2.56765334894647 5.92991384412531 \\
-2.54860126917726 5.89467832230026 \\
-2.54036531790085 5.87944649226042 \\
-2.51307728685523 5.82897914039552 \\
-2.49813391731237 5.80134239784323 \\
-2.48578925580961 5.77851178853062 \\
-2.45850122476399 5.72804443666573 \\
-2.44766656544747 5.70800647338619 \\
-2.43121319371837 5.67757708480083 \\
-2.40392516267275 5.62710973293594 \\
-2.39719921358257 5.61467054892916 \\
-2.37663713162713 5.57664238107104 \\
-2.34934910058151 5.52617502920614 \\
-2.34673186171768 5.52133462447212 \\
-2.32206106953589 5.47570767734125 \\
-2.29626450985278 5.42799870001508 \\
-2.29477303849027 5.42524032547635 \\
-2.26748500744465 5.37477297361145 \\
-2.24579715798788 5.33466277555805 \\
-2.24019697639903 5.32430562174656 \\
-2.2129089453534 5.27383826988166 \\
-2.19532980612299 5.24132685110101 \\
-2.18562091430778 5.22337091801676 \\
-2.15833288326216 5.17290356615187 \\
-2.14486245425809 5.14799092664398 \\
-2.13104485221654 5.12243621428697 \\
-2.10375682117092 5.07196886242208 \\
-2.09439510239319 5.05465500218694 \\
-2.0764687901253 5.02150151055718 \\
-2.04918075907968 4.97103415869228 \\
-2.0439277505283 4.96131907772991 \\
-2.02189272803406 4.92056680682739 \\
-1.99460469698844 4.87009945496249 \\
-1.9934603986634 4.86798315327287 \\
-1.96731666594282 4.81963210309759 \\
-1.94299304679851 4.77464722881584 \\
-1.9400286348972 4.7691647512327 \\
-1.91274060385158 4.7186973993678 \\
-1.89252569493361 4.6813113043588 \\
-1.88545257280596 4.6682300475029 \\
-1.85816454176034 4.61776269563801 \\
-1.84205834306871 4.58797537990177 \\
-1.83087651071472 4.56729534377311 \\
-1.8035884796691 4.51682799190822 \\
-1.79159099120382 4.49463945544473 \\
-1.77630044862348 4.46636064004332 \\
-1.74901241757786 4.41589328817842 \\
-1.74112363933892 4.40130353098769 \\
-1.72172438653224 4.36542593631353 \\
-1.69443635548662 4.31495858444863 \\
-1.69065628747403 4.30796760653066 \\
-1.667148324441 4.26449123258374 \\
-1.64018893560913 4.21463168207362 \\
-1.63986029339538 4.21402388071884 \\
-1.61257226234976 4.16355652885394 \\
-1.58972158374423 4.12129575761659 \\
-1.58528423130414 4.11308917698905 \\
-1.55799620025852 4.06262182512415 \\
-1.53925423187934 4.02795983315955 \\
-1.5307081692129 4.01215447325925 \\
-1.50342013816728 3.96168712139436 \\
-1.48878688001444 3.93462390870251 \\
-1.47613210712166 3.91121976952946 \\
-1.44884407607604 3.86075241766457 \\
-1.43831952814954 3.84128798424548 \\
-1.42155604503042 3.81028506579967 \\
-1.3942680139848 3.75981771393477 \\
-1.38785217628465 3.74795205978844 \\
-1.36697998293918 3.70935036206988 \\
-1.33969195189356 3.65888301020498 \\
-1.33738482441975 3.65461613533141 \\
-1.31240392084794 3.60841565834008 \\
-1.28691747255485 3.56128021087437 \\
-1.28511588980232 3.55794830647519 \\
-1.2578278587567 3.50748095461029 \\
-1.23645012068996 3.46794428641733 \\
-1.23053982771108 3.45701360274539 \\
-1.20325179666546 3.4065462508805 \\
-1.18598276882506 3.3746083619603 \\
-1.17596376561984 3.3560788990156 \\
-1.14867573457422 3.30561154715071 \\
-1.13551541696017 3.28127243750327 \\
-1.1213877035286 3.25514419528581 \\
-1.09409967248298 3.20467684342091 \\
-1.08504806509527 3.18793651304623 \\
-1.06681164143736 3.15420949155602 \\
-1.03952361039174 3.10374213969112 \\
-1.03458071323037 3.09460058858919 \\
-1.01223557934612 3.05327478782622 \\
-0.984947548300496 3.00280743596133 \\
-0.984113361365477 3.00126466413216 \\
-0.957659517254877 2.95234008409643 \\
-0.933646009500581 2.90792873967512 \\
-0.930371486209255 2.90187273223154 \\
-0.903083455163636 2.85140538036664 \\
-0.883178657635685 2.81459281521809 \\
-0.875795424118015 2.80093802850174 \\
-0.848507393072396 2.75047067663685 \\
-0.832711305770789 2.72125689076105 \\
-0.821219362026776 2.70000332477195 \\
-0.793931330981155 2.64953597290705 \\
-0.782243953905892 2.62792096630402 \\
-0.766643299935534 2.59906862104216 \\
-0.739355268889915 2.54860126917726 \\
-0.731776602040996 2.53458504184698 \\
-0.712067237844295 2.49813391731237 \\
-0.684779206798674 2.44766656544747 \\
-0.681309250176099 2.44124911738994 \\
-0.657491175753054 2.39719921358257 \\
-0.630841898311203 2.34791319293291 \\
-0.630203144707434 2.34673186171768 \\
-0.602915113661814 2.29626450985278 \\
-0.580374546446308 2.25457726847587 \\
-0.575627082616193 2.24579715798788 \\
-0.548339051570573 2.19532980612299 \\
-0.529907194581411 2.16124134401884 \\
-0.521051020524954 2.14486245425809 \\
-0.493762989479333 2.0943951023932 \\
-0.479439842716515 2.0679054195618 \\
-0.466474958433713 2.0439277505283 \\
-0.439186927388092 1.9934603986634 \\
-0.428972490851618 1.97456949510477 \\
-0.411898896342472 1.94299304679851 \\
-0.384610865296852 1.89252569493361 \\
-0.378505138986722 1.88123357064773 \\
-0.357322834251233 1.84205834306871 \\
-0.330034803205612 1.79159099120382 \\
-0.328037787121826 1.78789764619069 \\
-0.302746772159992 1.74112363933892 \\
-0.27757043525693 1.69456172173366 \\
-0.275458741114372 1.69065628747403 \\
-0.248170710068751 1.64018893560913 \\
-0.227103083392033 1.60122579727662 \\
-0.220882679023131 1.58972158374423 \\
-0.193594647977511 1.53925423187934 \\
-0.176635731527137 1.50788987281959 \\
-0.16630661693189 1.48878688001444 \\
-0.139018585886271 1.43831952814954 \\
-0.12616837966224 1.41455394836255 \\
-0.11173055484065 1.38785217628465 \\
-0.0844425237950302 1.33738482441975 \\
-0.0757010277973444 1.32121802390552 \\
-0.0571544927494096 1.28691747255485 \\
-0.0298664617037898 1.23645012068996 \\
-0.0252336759324487 1.22788209944848 \\
-0.00257843065816954 1.18598276882506 \\
0.0247096003874508 1.13551541696017 \\
0.0252336759324479 1.13454617499145 \\
0.0519976314330711 1.08504806509527 \\
0.0757010277973444 1.04121025053441 \\
0.0792856624786915 1.03458071323037 \\
0.106573693524311 0.984113361365478 \\
0.126168379662241 0.947874326077373 \\
0.133861724569931 0.933646009500581 \\
0.161149755615551 0.883178657635685 \\
0.176635731527137 0.854538401620339 \\
0.188437786661173 0.832711305770788 \\
0.215725817706792 0.782243953905892 \\
0.227103083392033 0.761202477163302 \\
0.243013848752413 0.731776602040996 \\
0.270301879798033 0.6813092501761 \\
0.277570435256929 0.667866552706268 \\
0.297589910843652 0.630841898311203 \\
0.324877941889273 0.580374546446308 \\
0.328037787121827 0.57453062824923 \\
0.352165972934893 0.52990719458141 \\
0.378505138986722 0.481194703792195 \\
0.379454003980513 0.479439842716515 \\
0.406742035026133 0.428972490851619 \\
0.428972490851619 0.387858779335159 \\
0.434030066071753 0.378505138986722 \\
0.461318097117373 0.328037787121827 \\
0.479439842716515 0.294522854878124 \\
0.488606128162994 0.277570435256929 \\
0.515894159208614 0.227103083392033 \\
0.52990719458141 0.20118693042109 \\
0.543182190254234 0.176635731527137 \\
0.570470221299854 0.126168379662241 \\
0.580374546446308 0.107851005964052 \\
0.597758252345474 0.0757010277973444 \\
0.625046283391095 0.0252336759324479 \\
0.630841898311203 0.0145150815070176 \\
0.652334314436716 -0.0252336759324487 \\
0.679622345482336 -0.0757010277973444 \\
0.6813092501761 -0.0788208429500183 \\
0.706910376527956 -0.12616837966224 \\
0.731776602040996 -0.172156767407053 \\
0.734198407573576 -0.176635731527137 \\
0.761486438619196 -0.227103083392033 \\
0.782243953905892 -0.26549269186409 \\
0.788774469664816 -0.27757043525693 \\
0.816062500710436 -0.328037787121826 \\
0.832711305770788 -0.358828616321124 \\
0.843350531756056 -0.378505138986722 \\
0.870638562801676 -0.428972490851618 \\
0.883178657635685 -0.452164540778161 \\
0.897926593847297 -0.479439842716515 \\
0.925214624892917 -0.529907194581411 \\
0.933646009500581 -0.545500465235196 \\
0.952502655938537 -0.580374546446308 \\
0.979790686984158 -0.630841898311203 \\
0.984113361365478 -0.638836389692233 \\
1.00707871802978 -0.681309250176099 \\
1.0343667490754 -0.731776602040996 \\
1.03458071323037 -0.732172314149267 \\
1.06165478012102 -0.782243953905892 \\
1.08504806509527 -0.825508238606302 \\
1.08894281116664 -0.832711305770789 \\
1.11623084221226 -0.883178657635685 \\
1.13551541696017 -0.91884416306334 \\
1.14351887325788 -0.933646009500581 \\
1.1708069043035 -0.984113361365477 \\
1.18598276882506 -1.01218008752037 \\
1.19809493534912 -1.03458071323037 \\
1.22538296639474 -1.08504806509527 \\
1.23645012068996 -1.10551601197741 \\
1.25267099744036 -1.13551541696017 \\
1.27995902848598 -1.18598276882506 \\
1.28691747255485 -1.19885193643445 \\
1.3072470595316 -1.23645012068996 \\
1.33453509057722 -1.28691747255485 \\
1.33738482441975 -1.29218786089148 \\
1.36182312162284 -1.33738482441975 \\
1.38785217628465 -1.38552378534852 \\
1.38911115266846 -1.38785217628465 \\
1.41639918371408 -1.43831952814954 \\
1.43831952814954 -1.47885970980555 \\
1.4436872147597 -1.48878688001444 \\
1.47097524580532 -1.53925423187934 \\
1.48878688001444 -1.57219563426259 \\
1.49826327685094 -1.58972158374423 \\
1.52555130789656 -1.64018893560913 \\
1.53925423187934 -1.66553155871962 \\
1.55283933894218 -1.69065628747403 \\
1.5801273699878 -1.74112363933892 \\
1.58972158374423 -1.75886748317666 \\
1.60741540103342 -1.79159099120382 \\
1.63470343207904 -1.84205834306871 \\
1.64018893560913 -1.85220340763369 \\
1.66199146312466 -1.89252569493361 \\
1.68927949417028 -1.94299304679851 \\
1.69065628747403 -1.94553933209073 \\
1.7165675252159 -1.9934603986634 \\
1.74112363933892 -2.03887525654777 \\
1.74385555626152 -2.0439277505283 \\
1.77114358730714 -2.09439510239319 \\
1.79159099120382 -2.1322111810048 \\
1.79843161835276 -2.14486245425809 \\
1.82571964939838 -2.19532980612299 \\
1.84205834306871 -2.22554710546184 \\
1.853007680444 -2.24579715798788 \\
1.88029571148962 -2.29626450985278 \\
1.89252569493361 -2.31888302991887 \\
1.90758374253524 -2.34673186171768 \\
1.93487177358086 -2.39719921358257 \\
1.94299304679851 -2.41221895437591 \\
1.96215980462649 -2.44766656544747 \\
1.98944783567211 -2.49813391731237 \\
1.9934603986634 -2.50555487883294 \\
2.01673586671772 -2.54860126917726 \\
2.0439277505283 -2.59889080328998 \\
2.04402389776335 -2.59906862104216 \\
2.07131192880896 -2.64953597290705 \\
2.0943951023932 -2.69222672774702 \\
2.09859995985459 -2.70000332477195 \\
2.12588799090021 -2.75047067663685 \\
2.14486245425809 -2.78556265220405 \\
2.15317602194582 -2.80093802850174 \\
2.18046405299145 -2.85140538036664 \\
2.19532980612299 -2.87889857666109 \\
2.20775208403707 -2.90187273223154 \\
2.23504011508269 -2.95234008409643 \\
2.24579715798788 -2.97223450111812 \\
2.26232814612831 -3.00280743596133 \\
2.28961617717393 -3.05327478782622 \\
2.29626450985278 -3.06557042557516 \\
2.31690420821955 -3.10374213969112 \\
2.34419223926517 -3.15420949155602 \\
2.34673186171768 -3.15890635003219 \\
2.37148027031079 -3.20467684342091 \\
2.39719921358257 -3.25224227448923 \\
2.39876830135641 -3.25514419528581 \\
2.42605633240203 -3.30561154715071 \\
2.44766656544747 -3.34557819894626 \\
2.45334436344765 -3.3560788990156 \\
2.48063239449327 -3.4065462508805 \\
2.49813391731237 -3.4389141234033 \\
2.50792042553889 -3.45701360274539 \\
2.53520845658451 -3.50748095461029 \\
2.54860126917726 -3.53225004786034 \\
2.56249648763013 -3.55794830647519 \\
2.58978451867575 -3.60841565834008 \\
2.59906862104216 -3.62558597231737 \\
2.61707254972137 -3.65888301020498 \\
2.64436058076699 -3.70935036206988 \\
2.64953597290705 -3.71892189677441 \\
2.67164861181261 -3.75981771393477 \\
2.69893664285823 -3.81028506579967 \\
2.70000332477195 -3.81225782123145 \\
2.72622467390385 -3.86075241766456 \\
2.75047067663685 -3.90559374568848 \\
2.75351270494947 -3.91121976952946 \\
2.78080073599509 -3.96168712139436 \\
2.80093802850174 -3.99892967014552 \\
2.80808876704071 -4.01215447325925 \\
2.83537679808633 -4.06262182512415 \\
2.85140538036664 -4.09226559460255 \\
2.86266482913195 -4.11308917698905 \\
2.88995286017757 -4.16355652885394 \\
2.90187273223154 -4.18560151905959 \\
2.91724089122319 -4.21402388071884 \\
2.94452892226881 -4.26449123258374 \\
2.95234008409643 -4.27893744351662 \\
2.97181695331443 -4.31495858444863 \\
2.99910498436005 -4.36542593631353 \\
3.00280743596133 -4.37227336797366 \\
3.02639301540567 -4.41589328817842 \\
3.05327478782622 -4.46560929243069 \\
3.05368104645129 -4.46636064004332 \\
3.08096907749691 -4.51682799190822 \\
3.10374213969112 -4.55894521688773 \\
3.10825710854253 -4.56729534377311 \\
3.13554513958815 -4.61776269563801 \\
3.15420949155602 -4.65228114134477 \\
3.16283317063377 -4.66823004750291 \\
3.19012120167939 -4.7186973993678 \\
3.20467684342091 -4.7456170658018 \\
3.21740923272501 -4.7691647512327 \\
3.24469726377063 -4.81963210309759 \\
3.25514419528581 -4.83895299025884 \\
3.27198529481625 -4.87009945496249 \\
3.29927332586187 -4.92056680682739 \\
3.30561154715071 -4.93228891471587 \\
3.32656135690749 -4.97103415869228 \\
3.35384938795311 -5.02150151055718 \\
3.3560788990156 -5.02562483917291 \\
3.38113741899873 -5.07196886242208 \\
3.4065462508805 -5.11896076362994 \\
3.40842545004436 -5.12243621428697 \\
3.43571348108997 -5.17290356615187 \\
3.45701360274539 -5.21229668808698 \\
3.46300151213559 -5.22337091801676 \\
3.49028954318121 -5.27383826988166 \\
3.50748095461029 -5.30563261254401 \\
3.51757757422684 -5.32430562174656 \\
3.54486560527246 -5.37477297361145 \\
3.55794830647519 -5.39896853700105 \\
3.57215363631807 -5.42524032547635 \\
3.5994416673637 -5.47570767734125 \\
3.60841565834008 -5.49230446145809 \\
3.62672969840932 -5.52617502920614 \\
3.65401772945494 -5.57664238107104 \\
3.65888301020498 -5.58564038591512 \\
3.68130576050056 -5.62710973293594 \\
3.70859379154618 -5.67757708480083 \\
3.70935036206988 -5.67897631037216 \\
3.7358818225918 -5.72804443666573 \\
3.75981771393477 -5.77231223482919 \\
3.76316985363742 -5.77851178853062 \\
3.79045788468304 -5.82897914039552 \\
3.81028506579967 -5.86564815928623 \\
3.81774591572866 -5.87944649226042 \\
3.84503394677428 -5.92991384412531 \\
3.86075241766457 -5.95898408374327 \\
3.8723219778199 -5.98038119599021 \\
3.89961000886552 -6.03084854785511 \\
3.91121976952946 -6.0523200082003 \\
3.92689803991114 -6.08131589972 \\
3.95418607095676 -6.1317832515849 \\
3.96168712139436 -6.14565593265733 \\
3.98147410200238 -6.18225060344979 \\
4.008762133048 -6.23271795531469 \\
4.01215447325925 -6.23899185711437 \\
4.03605016409362 -6.28318530717959 \\
};

\addlegendentry{decision boundary};

\addplot [
color=green,
only marks,
mark=asterisk,
mark options={solid}
]
table[row sep=crcr]{
0.39412761 0.94238725\\
0.50301449 0.33508338\\
0.72197985 0.43736382\\
0.46676255 0.53249823\\
0.66405187 0.39870305\\
0.72406171 0.35841852\\
0.26181868 0.86863524\\
0.7084714 0.62641267\\
0.78385902 0.24117231\\
0.98615781 0.97808165\\
0.47334271 0.64050078\\
0.90281883 0.22984865\\
0.45105876 0.68133515\\
0.80451681 0.66582341\\
0.82886448 0.13471797\\
0.71812397 0.069318238\\
0.56918952 0.8529305\\
0.44348322 0.53651682\\
0.8518447 0.012886306\\
0.75947939 0.88920608\\
0.94975928 0.8660206\\
0.55793851 0.25424693\\
0.59617708 0.15926482\\
0.81620571 0.59436442\\
0.97709235 0.33110008\\
0.70368367 0.86363422\\
0.52206092 0.56762331\\
0.93289706 0.98048127\\
0.71335444 0.7918316\\
0.4496421 0.83302723\\
0.96882014 0.63898657\\
0.3557161 0.66900008\\
0.75533857 0.37981775\\
0.89481276 0.44158549\\
0.93273619 0.17599564\\
0.94081954 0.790224\\
0.70185317 0.5136085\\
0.84767647 0.21322938\\
0.8511224 0.40775702\\
0.3192963 0.94181515\\
0.8677957 0.38437405\\
0.199838 0.89664815\\
0.56670978 0.73399632\\
0.52211176 0.39979385\\
0.76991848 0.16930586\\
0.3750558 0.52474549\\
0.8233872 0.64120266\\
0.59791326 0.83685156\\
0.94915039 0.80346219\\
0.28879765 0.69778483\\
0.88883261 0.46188778\\
0.93309842 0.44535473\\
0.99953167 0.87535098\\
0.21198796 0.8352594\\
0.4984098 0.33309507\\
0.2904883 0.88070533\\
0.67275441 0.47968681\\
0.95799112 0.56081673\\
0.76655153 0.61590866\\
0.66612374 0.66189901\\
0.28819335 0.71396084\\
0.81673121 0.5152079\\
0.98548351 0.60586579\\
0.81939293 0.8221174\\
0.62113871 0.31775068\\
0.56022204 0.58769677\\
0.82200759 0.25435426\\
0.26321193 0.8030309\\
0.75363453 0.66784589\\
0.6596448 0.013626353\\
0.60211691 0.45456094\\
0.60493711 0.90494922\\
0.65950155 0.28215885\\
0.63654691 0.47659189\\
0.17030911 0.98371192\\
0.53960111 0.92234958\\
0.62339418 0.56119625\\
0.68588997 0.65232469\\
0.67734595 0.7726796\\
0.87683002 0.10617579\\
0.77907925 0.0068577917\\
0.92667831 0.19566205\\
0.67871988 0.78714313\\
0.22715372 0.76170319\\
0.516252 0.90703504\\
0.45820422 0.7585686\\
0.70320334 0.38072994\\
0.58248388 0.33111079\\
0.509206 0.50407844\\
0.37960413 0.77986778\\
0.77088184 0.80221322\\
0.63819322 0.2027589\\
0.98656686 0.57961489\\
0.5028758 0.66650026\\
0.94770338 0.67676529\\
0.82802584 0.94251124\\
0.9175572 0.77014853\\
0.81212591 0.86626227\\
0.90826256 0.99094832\\
0.76266647 0.79261032\\
0.72180028 0.44864921\\
0.65163982 0.52435711\\
0.75402267 0.17147322\\
0.66316201 0.13066575\\
0.8834935 0.21878112\\
0.85056678 0.22478673\\
0.34020345 0.90887473\\
0.91376324 0.58873933\\
0.8620449 0.65352398\\
0.65661854 0.31343498\\
0.8911828 0.23115882\\
0.48814352 0.41606383\\
0.99264565 0.29879879\\
0.37332594 0.67243639\\
0.53137832 0.93825748\\
0.50194434 0.56296257\\
0.6604274 0.16902126\\
0.67365301 0.27889542\\
0.95733018 0.55681466\\
0.56505387 0.23192084\\
0.96916627 0.47865839\\
0.87021582 0.79272079\\
0.51952869 0.90960045\\
0.19229142 0.92219639\\
0.71568905 0.013266366\\
0.25067278 0.76754963\\
0.93386482 0.94734264\\
0.52162234 0.92382979\\
0.89520169 0.19899726\\
};
\addlegendentry{class 1};

\addplot [
color=red,
only marks,
mark=+,
mark options={solid}
]
table[row sep=crcr]{
0.30620855 0.47115591\\
0.11216371 0.14931034\\
0.44328996 0.13586439\\
0.014668875 0.72578931\\
0.2816336 0.28527941\\
0.16627012 0.022493293\\
0.39390601 0.26219945\\
0.52075748 0.11651516\\
0.46080617 0.18033067\\
0.44530705 0.032418564\\
0.087744606 0.73392626\\
0.36629985 0.27602966\\
0.30253382 0.36845815\\
0.014233016 0.5694806\\
0.22190808 0.65861265\\
0.2280389 0.15259362\\
0.1721997 0.19186326\\
0.049046828 0.7720878\\
0.28614965 0.48305999\\
0.25120055 0.60810566\\
0.13098247 0.002025561\\
0.20927164 0.1034498\\
0.45509169 0.15733667\\
0.081073725 0.40751498\\
0.56204868 0.052692688\\
0.37489926 0.14997168\\
0.37217624 0.31105856\\
0.073690056 0.16853444\\
0.04949328 0.32272443\\
0.12192475 0.41090418\\
0.11706015 0.50552221\\
0.046636146 0.016197484\\
0.10158548 0.082612618\\
0.065314507 0.82071701\\
0.23429973 0.19302002\\
0.063127896 0.01295783\\
0.26421766 0.3087418\\
0.13094482 0.61663323\\
0.095413016 0.68514\\
0.014864337 0.51015303\\
0.017362693 0.96670263\\
0.24403153 0.13020167\\
0.21406286 0.56157937\\
0.18336363 0.065034354\\
0.0128905 0.0010733656\\
0.31040223 0.54176379\\
0.30729608 0.45133767\\
0.074321493 0.61856262\\
0.070669191 0.015520957\\
0.011930467 0.89085396\\
0.074289609 0.56456859\\
0.19323631 0.76719736\\
0.27643472 0.48409788\\
0.31392999 0.47101215\\
0.11307983 0.73740432\\
0.15637612 0.5039276\\
0.12211877 0.62908555\\
0.27215744 0.1054805\\
0.41943272 0.14142585\\
0.21299366 0.4569683\\
0.035599967 0.78813305\\
0.081163704 0.28106398\\
0.46615498 0.0073288452\\
0.22857671 0.54211787\\
0.18131633 0.34314769\\
0.42219483 0.11888856\\
0.19186575 0.48558916\\
0.11121644 0.95222323\\
0.023743871 0.5265224\\
0.026876593 0.19300765\\
0.13718931 0.81330552\\
};
\addlegendentry{class -1};

\end{axis}
\end{tikzpicture}%
	\caption{Decision boundary for target data set 1 with 10 epochs}
	\label{fig:decisionBoundary1}
\end{figure}

\begin{figure}
	\centering
	\setlength\figureheight{7cm} 
	\setlength\figurewidth{9cm}
	% This file was created by matlab2tikz v0.4.4 running on MATLAB 8.0.
% Copyright (c) 2008--2013, Nico Schlömer <nico.schloemer@gmail.com>
% All rights reserved.
% 
% The latest updates can be retrieved from
%   http://www.mathworks.com/matlabcentral/fileexchange/22022-matlab2tikz
% where you can also make suggestions and rate matlab2tikz.
% 
%
% defining custom colors
\definecolor{mycolor1}{rgb}{1,0,1}%
%
\begin{tikzpicture}

\begin{axis}[%
width=\figurewidth,
height=\figureheight,
colormap/jet,
unbounded coords=jump,
scale only axis,
xmin=-0.0880227,
xmax=1.099484837,
xlabel={${\text{x}}$},
ymin=-0.0988798014,
ymax=1.090901487,
ylabel={${\text{y}}$},
title={Decision boundary for target data set 2 with 10 epochs},
legend style={draw=black,fill=white,legend cell align=left}
]

\addplot[area legend,solid,draw=mycolor1]
table[row sep=crcr]{
x y\\
-4.54530800417533 6.28318530717959 \\
-4.51682799190822 6.25113025361257 \\
-4.50046918345097 6.23271795531469 \\
-4.46636064004332 6.19432783189161 \\
-4.45563036272663 6.18225060344979 \\
-4.41589328817842 6.13752541017064 \\
-4.41079154200228 6.1317832515849 \\
-4.36595272127793 6.08131589972 \\
-4.36542593631353 6.08072298844968 \\
-4.32111390055358 6.03084854785511 \\
-4.31495858444863 6.02392056672871 \\
-4.27627507982923 5.98038119599021 \\
-4.26449123258374 5.96711814500775 \\
-4.23143625910488 5.92991384412531 \\
-4.21402388071884 5.91031572328678 \\
-4.18659743838053 5.87944649226042 \\
-4.16355652885394 5.85351330156582 \\
-4.14175861765618 5.82897914039552 \\
-4.11308917698905 5.79671087984485 \\
-4.09691979693183 5.77851178853062 \\
-4.06262182512415 5.73990845812389 \\
-4.05208097620748 5.72804443666573 \\
-4.01215447325925 5.68310603640292 \\
-4.00724215548313 5.67757708480083 \\
-3.96240333475878 5.62710973293594 \\
-3.96168712139436 5.62630361468195 \\
-3.91756451403443 5.57664238107104 \\
-3.91121976952946 5.56950119296099 \\
-3.87272569331008 5.52617502920614 \\
-3.86075241766456 5.51269877124002 \\
-3.82788687258574 5.47570767734125 \\
-3.81028506579967 5.45589634951906 \\
-3.78304805186139 5.42524032547635 \\
-3.75981771393477 5.39909392779809 \\
-3.73820923113704 5.37477297361145 \\
-3.70935036206988 5.34229150607713 \\
-3.69337041041269 5.32430562174656 \\
-3.65888301020498 5.28548908435616 \\
-3.64853158968834 5.27383826988166 \\
-3.60841565834008 5.2286866626352 \\
-3.60369276896399 5.22337091801676 \\
-3.55885394823964 5.17290356615187 \\
-3.55794830647519 5.17188424091423 \\
-3.51401512751529 5.12243621428697 \\
-3.50748095461029 5.11508181919326 \\
-3.46917630679094 5.07196886242208 \\
-3.45701360274539 5.0582793974723 \\
-3.42433748606659 5.02150151055718 \\
-3.4065462508805 5.00147697575134 \\
-3.37949866534224 4.97103415869228 \\
-3.3560788990156 4.94467455403037 \\
-3.3346598446179 4.92056680682739 \\
-3.30561154715071 4.8878721323094 \\
-3.28982102389355 4.87009945496249 \\
-3.25514419528581 4.83106971058844 \\
-3.2449822031692 4.81963210309759 \\
-3.20467684342091 4.77426728886747 \\
-3.20014338244485 4.7691647512327 \\
-3.1553045617205 4.7186973993678 \\
-3.15420949155602 4.71746486714651 \\
-3.11046574099615 4.6682300475029 \\
-3.10374213969112 4.66066244542554 \\
-3.0656269202718 4.61776269563801 \\
-3.05327478782622 4.60386002370458 \\
-3.02078809954745 4.56729534377311 \\
-3.00280743596133 4.54705760198361 \\
-2.9759492788231 4.51682799190822 \\
-2.95234008409643 4.49025518026265 \\
-2.93111045809875 4.46636064004332 \\
-2.90187273223154 4.43345275854168 \\
-2.8862716373744 4.41589328817842 \\
-2.85140538036664 4.37665033682072 \\
-2.84143281665005 4.36542593631353 \\
-2.80093802850174 4.31984791509975 \\
-2.7965939959257 4.31495858444863 \\
-2.75175517520135 4.26449123258374 \\
-2.75047067663685 4.26304549337879 \\
-2.706916354477 4.21402388071884 \\
-2.70000332477195 4.20624307165782 \\
-2.66207753375265 4.16355652885394 \\
-2.64953597290705 4.14944064993685 \\
-2.6172387130283 4.11308917698905 \\
-2.59906862104216 4.09263822821589 \\
-2.57239989230396 4.06262182512415 \\
-2.54860126917726 4.03583580649492 \\
-2.52756107157961 4.01215447325925 \\
-2.49813391731237 3.97903338477396 \\
-2.48272225085526 3.96168712139436 \\
-2.44766656544747 3.922230963053 \\
-2.43788343013091 3.91121976952946 \\
-2.39719921358257 3.86542854133203 \\
-2.39304460940656 3.86075241766457 \\
-2.34820578868221 3.81028506579967 \\
-2.34673186171768 3.80862611961106 \\
-2.30336696795786 3.75981771393477 \\
-2.29626450985278 3.7518236978901 \\
-2.25852814723351 3.70935036206988 \\
-2.24579715798788 3.69502127616913 \\
-2.21368932650916 3.65888301020498 \\
-2.19532980612299 3.63821885444817 \\
-2.16885050578481 3.60841565834008 \\
-2.14486245425809 3.5814164327272 \\
-2.12401168506046 3.55794830647519 \\
-2.09439510239319 3.52461401100624 \\
-2.07917286433611 3.50748095461029 \\
-2.0439277505283 3.46781158928527 \\
-2.03433404361176 3.45701360274539 \\
-1.9934603986634 3.4110091675643 \\
-1.98949522288742 3.4065462508805 \\
-1.94465640216307 3.3560788990156 \\
-1.94299304679851 3.35420674584334 \\
-1.89981758143872 3.30561154715071 \\
-1.89252569493361 3.29740432412238 \\
-1.85497876071437 3.25514419528581 \\
-1.84205834306871 3.24060190240141 \\
-1.81013993999002 3.20467684342091 \\
-1.79159099120382 3.18379948068044 \\
-1.76530111926567 3.15420949155602 \\
-1.74112363933892 3.12699705895948 \\
-1.72046229854132 3.10374213969112 \\
-1.69065628747403 3.07019463723851 \\
-1.67562347781697 3.05327478782622 \\
-1.64018893560913 3.01339221551755 \\
-1.63078465709262 3.00280743596133 \\
-1.58972158374423 2.95658979379658 \\
-1.58594583636827 2.95234008409643 \\
-1.54110701564392 2.90187273223154 \\
-1.53925423187934 2.89978737207562 \\
-1.49626819491957 2.85140538036664 \\
-1.48878688001444 2.84298495035465 \\
-1.45142937419522 2.80093802850174 \\
-1.43831952814954 2.78618252863369 \\
-1.40659055347088 2.75047067663685 \\
-1.38785217628465 2.72938010691272 \\
-1.36175173274653 2.70000332477195 \\
-1.33738482441975 2.67257768519176 \\
-1.31691291202218 2.64953597290705 \\
-1.28691747255485 2.61577526347079 \\
-1.27207409129782 2.59906862104216 \\
-1.23645012068996 2.55897284174983 \\
-1.22723527057348 2.54860126917726 \\
-1.18598276882506 2.50217042002886 \\
-1.18239644984913 2.49813391731237 \\
-1.13755762912478 2.44766656544747 \\
-1.13551541696017 2.4453679983079 \\
-1.09271880840043 2.39719921358257 \\
-1.08504806509527 2.38856557658693 \\
-1.04787998767608 2.34673186171768 \\
-1.03458071323037 2.33176315486596 \\
-1.00304116695173 2.29626450985278 \\
-0.984113361365477 2.274960733145 \\
-0.958202346227381 2.24579715798788 \\
-0.933646009500581 2.21815831142403 \\
-0.913363525503032 2.19532980612299 \\
-0.883178657635685 2.16135588970307 \\
-0.868524704778683 2.14486245425809 \\
-0.832711305770789 2.1045534679821 \\
-0.823685884054334 2.0943951023932 \\
-0.782243953905892 2.04775104626114 \\
-0.778847063329984 2.0439277505283 \\
-0.734008242605633 1.9934603986634 \\
-0.731776602040996 1.99094862454017 \\
-0.689169421881285 1.94299304679851 \\
-0.681309250176099 1.93414620281921 \\
-0.644330601156936 1.89252569493361 \\
-0.630841898311203 1.87734378109824 \\
-0.599491780432588 1.84205834306871 \\
-0.580374546446308 1.82054135937728 \\
-0.554652959708237 1.79159099120382 \\
-0.529907194581411 1.76373893765631 \\
-0.509814138983888 1.74112363933892 \\
-0.479439842716515 1.70693651593534 \\
-0.464975318259539 1.69065628747403 \\
-0.428972490851618 1.65013409421438 \\
-0.420136497535189 1.64018893560913 \\
-0.378505138986722 1.59333167249341 \\
-0.37529767681084 1.58972158374423 \\
-0.330458856086491 1.53925423187934 \\
-0.328037787121826 1.53652925077245 \\
-0.285620035362142 1.48878688001444 \\
-0.27757043525693 1.47972682905148 \\
-0.240781214637793 1.43831952814954 \\
-0.227103083392033 1.42292440733052 \\
-0.195942393913443 1.38785217628465 \\
-0.176635731527137 1.36612198560955 \\
-0.151103573189094 1.33738482441975 \\
-0.12616837966224 1.30931956388859 \\
-0.106264752464744 1.28691747255485 \\
-0.0757010277973444 1.25251714216762 \\
-0.0614259317403959 1.23645012068996 \\
-0.0252336759324487 1.19571472044666 \\
-0.0165871110160465 1.18598276882506 \\
0.0252336759324479 1.13891229872569 \\
0.0282517097083028 1.13551541696017 \\
0.0730905304326529 1.08504806509527 \\
0.0757010277973444 1.08210987700473 \\
0.117929351157002 1.03458071323037 \\
0.126168379662241 1.02530745528376 \\
0.162768171881351 0.984113361365478 \\
0.176635731527137 0.968505033562796 \\
0.2076069926057 0.933646009500581 \\
0.227103083392033 0.91170261184183 \\
0.252445813330049 0.883178657635685 \\
0.277570435256929 0.854900190120866 \\
0.2972846340544 0.832711305770788 \\
0.328037787121827 0.798097768399899 \\
0.342123454778748 0.782243953905892 \\
0.378505138986722 0.741295346678935 \\
0.386962275503098 0.731776602040996 \\
0.428972490851619 0.684492924957968 \\
0.431801096227447 0.6813092501761 \\
0.476639916951796 0.630841898311203 \\
0.479439842716515 0.627690503237004 \\
0.521478737676144 0.580374546446308 \\
0.52990719458141 0.570888081516039 \\
0.566317558400496 0.52990719458141 \\
0.580374546446308 0.514085659795072 \\
0.611156379124845 0.479439842716515 \\
0.630841898311203 0.457283238074108 \\
0.655995199849192 0.428972490851619 \\
0.6813092501761 0.400480816353142 \\
0.700834020573543 0.378505138986722 \\
0.731776602040996 0.343678394632177 \\
0.745672841297891 0.328037787121827 \\
0.782243953905892 0.286875972911212 \\
0.790511662022242 0.277570435256929 \\
0.832711305770788 0.230073551190247 \\
0.835350482746591 0.227103083392033 \\
0.880189303470939 0.176635731527137 \\
0.883178657635685 0.17327112946928 \\
0.925028124195288 0.126168379662241 \\
0.933646009500581 0.116468707748315 \\
0.969866944919639 0.0757010277973444 \\
0.984113361365478 0.0596662860273496 \\
1.01470576564399 0.0252336759324479 \\
1.03458071323037 0.00286386430638513 \\
1.05954458636834 -0.0252336759324487 \\
1.08504806509527 -0.0539385574145795 \\
1.10438340709269 -0.0757010277973444 \\
1.13551541696017 -0.110740979135546 \\
1.14922222781704 -0.12616837966224 \\
1.18598276882506 -0.167543400856511 \\
1.19406104854139 -0.176635731527137 \\
1.23645012068996 -0.224345822577477 \\
1.23889986926573 -0.227103083392033 \\
1.28373868999008 -0.27757043525693 \\
1.28691747255485 -0.281148244298442 \\
1.32857751071443 -0.328037787121826 \\
1.33738482441975 -0.337950666019407 \\
1.37341633143878 -0.378505138986722 \\
1.38785217628465 -0.394753087740372 \\
1.41825515216313 -0.428972490851618 \\
1.43831952814954 -0.451555509461338 \\
1.46309397288748 -0.479439842716515 \\
1.48878688001444 -0.508357931182303 \\
1.50793279361183 -0.529907194581411 \\
1.53925423187934 -0.565160352903268 \\
1.55277161433618 -0.580374546446308 \\
1.58972158374423 -0.621962774624234 \\
1.59761043506053 -0.630841898311203 \\
1.64018893560913 -0.678765196345199 \\
1.64244925578488 -0.681309250176099 \\
1.68728807650923 -0.731776602040996 \\
1.69065628747403 -0.735567618066165 \\
1.73212689723358 -0.782243953905892 \\
1.74112363933892 -0.79237003978713 \\
1.77696571795793 -0.832711305770789 \\
1.79159099120382 -0.849172461508094 \\
1.82180453868227 -0.883178657635685 \\
1.84205834306871 -0.905974883229061 \\
1.86664335940662 -0.933646009500581 \\
1.89252569493361 -0.962777304950026 \\
1.91148218013097 -0.984113361365477 \\
1.94299304679851 -1.01957972667099 \\
1.95632100085532 -1.03458071323037 \\
1.9934603986634 -1.07638214839196 \\
2.00115982157967 -1.08504806509527 \\
2.0439277505283 -1.13318457011292 \\
2.04599864230402 -1.13551541696017 \\
2.09083746302837 -1.18598276882506 \\
2.0943951023932 -1.18998699183389 \\
2.13567628375272 -1.23645012068996 \\
2.14486245425809 -1.24678941355485 \\
2.18051510447707 -1.28691747255485 \\
2.19532980612299 -1.30359183527582 \\
2.22535392520142 -1.33738482441975 \\
2.24579715798788 -1.36039425699678 \\
2.27019274592577 -1.38785217628465 \\
2.29626450985278 -1.41719667871775 \\
2.31503156665012 -1.43831952814954 \\
2.34673186171768 -1.47399910043871 \\
2.35987038737447 -1.48878688001444 \\
2.39719921358257 -1.53080152215968 \\
2.40470920809882 -1.53925423187934 \\
2.44766656544747 -1.58760394388064 \\
2.44954802882316 -1.58972158374423 \\
2.49438684954752 -1.64018893560913 \\
2.49813391731237 -1.64440636560161 \\
2.53922567027186 -1.69065628747403 \\
2.54860126917726 -1.70120878732258 \\
2.58406449099621 -1.74112363933892 \\
2.59906862104216 -1.75801120904354 \\
2.62890331172056 -1.79159099120382 \\
2.64953597290705 -1.81481363076451 \\
2.67374213244491 -1.84205834306871 \\
2.70000332477195 -1.87161605248547 \\
2.71858095316926 -1.89252569493361 \\
2.75047067663685 -1.92841847420644 \\
2.76341977389361 -1.94299304679851 \\
2.80093802850174 -1.9852208959274 \\
2.80825859461796 -1.9934603986634 \\
2.85140538036664 -2.04202331764837 \\
2.85309741534231 -2.0439277505283 \\
2.89793623606666 -2.09439510239319 \\
2.90187273223154 -2.09882573936933 \\
2.94277505679101 -2.14486245425809 \\
2.95234008409643 -2.1556281610903 \\
2.98761387751536 -2.19532980612299 \\
3.00280743596133 -2.21243058281126 \\
3.03245269823971 -2.24579715798788 \\
3.05327478782622 -2.26923300453223 \\
3.07729151896405 -2.29626450985278 \\
3.10374213969112 -2.32603542625319 \\
3.1221303396884 -2.34673186171768 \\
3.15420949155602 -2.38283784797416 \\
3.16696916041275 -2.39719921358257 \\
3.20467684342091 -2.43964026969513 \\
3.2118079811371 -2.44766656544747 \\
3.25514419528581 -2.49644269141609 \\
3.25664680186145 -2.49813391731237 \\
3.3014856225858 -2.54860126917726 \\
3.30561154715071 -2.55324511313706 \\
3.34632444331015 -2.59906862104216 \\
3.3560788990156 -2.61004753485802 \\
3.3911632640345 -2.64953597290705 \\
3.4065462508805 -2.66684995657899 \\
3.43600208475885 -2.70000332477195 \\
3.45701360274539 -2.72365237829995 \\
3.4808409054832 -2.75047067663685 \\
3.50748095461029 -2.78045480002092 \\
3.52567972620755 -2.80093802850174 \\
3.55794830647519 -2.83725722174188 \\
3.5705185469319 -2.85140538036664 \\
3.60841565834008 -2.89405964346285 \\
3.61535736765625 -2.90187273223154 \\
3.65888301020498 -2.95086206518381 \\
3.6601961883806 -2.95234008409643 \\
3.70503500910495 -3.00280743596133 \\
3.70935036206988 -3.00766448690478 \\
3.7498738298293 -3.05327478782622 \\
3.75981771393477 -3.06446690862574 \\
3.79471265055364 -3.10374213969112 \\
3.81028506579967 -3.12126933034671 \\
3.83955147127799 -3.15420949155602 \\
3.86075241766457 -3.17807175206768 \\
3.88439029200234 -3.20467684342091 \\
3.91121976952946 -3.23487417378864 \\
3.92922911272669 -3.25514419528581 \\
3.96168712139436 -3.2916765955096 \\
3.97406793345104 -3.30561154715071 \\
4.01215447325925 -3.34847901723057 \\
4.01890675417539 -3.3560788990156 \\
4.06262182512415 -3.40528143895154 \\
4.06374557489974 -3.4065462508805 \\
4.10858439562409 -3.45701360274539 \\
4.11308917698905 -3.4620838606725 \\
4.15342321634844 -3.50748095461029 \\
4.16355652885394 -3.51888628239346 \\
4.19826203707279 -3.55794830647519 \\
4.21402388071884 -3.57568870411443 \\
4.24310085779714 -3.60841565834008 \\
4.26449123258374 -3.6324911258354 \\
4.28793967852149 -3.65888301020498 \\
4.31495858444863 -3.68929354755636 \\
4.33277849924584 -3.70935036206988 \\
4.36542593631353 -3.74609596927733 \\
4.37761731997018 -3.75981771393477 \\
4.41589328817842 -3.80289839099829 \\
4.42245614069453 -3.81028506579967 \\
4.46636064004332 -3.85970081271926 \\
4.46729496141888 -3.86075241766456 \\
4.51213378214323 -3.91121976952946 \\
4.51682799190822 -3.91650323444022 \\
4.55697260286758 -3.96168712139436 \\
4.56729534377311 -3.97330565616119 \\
4.60181142359193 -4.01215447325925 \\
4.61776269563801 -4.03010807788215 \\
4.64665024431628 -4.06262182512415 \\
4.6682300475029 -4.08691049960312 \\
4.69148906504063 -4.11308917698905 \\
4.7186973993678 -4.14371292132409 \\
4.73632788576498 -4.16355652885394 \\
4.7691647512327 -4.20051534304505 \\
4.78116670648933 -4.21402388071884 \\
4.81963210309759 -4.25731776476602 \\
4.82600552721368 -4.26449123258374 \\
4.87009945496249 -4.31412018648698 \\
4.87084434793803 -4.31495858444863 \\
4.91568316866238 -4.36542593631353 \\
4.92056680682739 -4.37092260820795 \\
4.96052198938673 -4.41589328817842 \\
4.97103415869228 -4.42772502992891 \\
5.00536081011108 -4.46636064004332 \\
5.02150151055718 -4.48452745164988 \\
5.05019963083542 -4.51682799190822 \\
5.07196886242208 -4.54132987337084 \\
5.09503845155977 -4.56729534377311 \\
5.12243621428697 -4.59813229509181 \\
5.13987727228412 -4.61776269563801 \\
5.17290356615187 -4.65493471681277 \\
5.18471609300847 -4.66823004750291 \\
5.22337091801676 -4.71173713853374 \\
5.22955491373282 -4.7186973993678 \\
5.27383826988166 -4.7685395602547 \\
5.27439373445717 -4.7691647512327 \\
5.31923255518152 -4.81963210309759 \\
5.32430562174656 -4.82534198197567 \\
5.36407137590587 -4.87009945496249 \\
5.37477297361145 -4.88214440369664 \\
5.40891019663022 -4.92056680682739 \\
5.42524032547635 -4.9389468254176 \\
5.45374901735457 -4.97103415869228 \\
5.47570767734125 -4.99574924713857 \\
5.49858783807892 -5.02150151055718 \\
5.52617502920614 -5.05255166885953 \\
5.54342665880327 -5.07196886242208 \\
5.57664238107104 -5.10935409058049 \\
5.58826547952762 -5.12243621428697 \\
5.62710973293594 -5.16615651230146 \\
5.63310430025197 -5.17290356615187 \\
5.67757708480083 -5.22295893402243 \\
5.67794312097631 -5.22337091801676 \\
5.72278194170066 -5.27383826988166 \\
5.72804443666573 -5.27976135574339 \\
5.76762076242501 -5.32430562174656 \\
5.77851178853062 -5.33656377746436 \\
5.81245958314936 -5.37477297361145 \\
5.82897914039552 -5.39336619918532 \\
5.85729840387371 -5.42524032547635 \\
5.87944649226042 -5.45016862090629 \\
5.90213722459806 -5.47570767734125 \\
5.92991384412531 -5.50697104262725 \\
5.94697604532241 -5.52617502920614 \\
5.98038119599021 -5.56377346434822 \\
5.99181486604676 -5.57664238107104 \\
6.03084854785511 -5.62057588606918 \\
6.03665368677111 -5.62710973293594 \\
6.08131589972 -5.67737830779015 \\
6.08149250749546 -5.67757708480083 \\
6.12633132821981 -5.72804443666573 \\
6.1317832515849 -5.73418072951111 \\
6.17117014894416 -5.77851178853062 \\
6.18225060344979 -5.79098315123208 \\
6.21600896966851 -5.82897914039552 \\
6.23271795531469 -5.84778557295304 \\
6.26084779039286 -5.87944649226042 \\
6.28318530717959 -5.90458799467401 \\
NaN NaN \\
};

\addlegendentry{decision boundary};

\addplot [
color=green,
only marks,
mark=asterisk,
mark options={solid}
]
table[row sep=crcr]{
0.39412761 0.94238725\\
0.72197985 0.43736382\\
0.46676255 0.53249823\\
0.66405187 0.39870305\\
0.26181868 0.86863524\\
0.7084714 0.62641267\\
0.78385902 0.24117231\\
0.98615781 0.97808165\\
0.47334271 0.64050078\\
0.90281883 0.22984865\\
0.45105876 0.68133515\\
0.80451681 0.66582341\\
0.52075748 0.11651516\\
0.71812397 0.069318238\\
0.56918952 0.8529305\\
0.44348322 0.53651682\\
0.8518447 0.012886306\\
0.75947939 0.88920608\\
0.94975928 0.8660206\\
0.55793851 0.25424693\\
0.59617708 0.15926482\\
0.81620571 0.59436442\\
0.97709235 0.33110008\\
0.70368367 0.86363422\\
0.52206092 0.56762331\\
0.93289706 0.98048127\\
0.71335444 0.7918316\\
0.4496421 0.83302723\\
0.96882014 0.63898657\\
0.3557161 0.66900008\\
0.75533857 0.37981775\\
0.93273619 0.17599564\\
0.94081954 0.790224\\
0.70185317 0.5136085\\
0.45509169 0.15733667\\
0.8511224 0.40775702\\
0.3192963 0.94181515\\
0.199838 0.89664815\\
0.56670978 0.73399632\\
0.52211176 0.39979385\\
0.76991848 0.16930586\\
0.3750558 0.52474549\\
0.8233872 0.64120266\\
0.94915039 0.80346219\\
0.28879765 0.69778483\\
0.88883261 0.46188778\\
0.93309842 0.44535473\\
0.063127896 0.01295783\\
0.99953167 0.87535098\\
0.21198796 0.8352594\\
0.4984098 0.33309507\\
0.2904883 0.88070533\\
0.67275441 0.47968681\\
0.95799112 0.56081673\\
0.76655153 0.61590866\\
0.66612374 0.66189901\\
0.28819335 0.71396084\\
0.81673121 0.5152079\\
0.98548351 0.60586579\\
0.81939293 0.8221174\\
0.62113871 0.31775068\\
0.56022204 0.58769677\\
0.82200759 0.25435426\\
0.26321193 0.8030309\\
0.75363453 0.66784589\\
0.6596448 0.013626353\\
0.60211691 0.45456094\\
0.65950155 0.28215885\\
0.63654691 0.47659189\\
0.17030911 0.98371192\\
0.53960111 0.92234958\\
0.62339418 0.56119625\\
0.67734595 0.7726796\\
0.87683002 0.10617579\\
0.77907925 0.0068577917\\
0.92667831 0.19566205\\
0.67871988 0.78714313\\
0.22715372 0.76170319\\
0.516252 0.90703504\\
0.45820422 0.7585686\\
0.70320334 0.38072994\\
0.58248388 0.33111079\\
0.19323631 0.76719736\\
0.37960413 0.77986778\\
0.77088184 0.80221322\\
0.63819322 0.2027589\\
0.98656686 0.57961489\\
0.5028758 0.66650026\\
0.82802584 0.94251124\\
0.9175572 0.77014853\\
0.81212591 0.86626227\\
0.90826256 0.99094832\\
0.76266647 0.79261032\\
0.72180028 0.44864921\\
0.65163982 0.52435711\\
0.75402267 0.17147322\\
0.66316201 0.13066575\\
0.8834935 0.21878112\\
0.34020345 0.90887473\\
0.8620449 0.65352398\\
0.65661854 0.31343498\\
0.8911828 0.23115882\\
0.48814352 0.41606383\\
0.99264565 0.29879879\\
0.37332594 0.67243639\\
0.53137832 0.93825748\\
0.18131633 0.34314769\\
0.50194434 0.56296257\\
0.6604274 0.16902126\\
0.67365301 0.27889542\\
0.95733018 0.55681466\\
0.56505387 0.23192084\\
0.87021582 0.79272079\\
0.51952869 0.90960045\\
0.19229142 0.92219639\\
0.71568905 0.013266366\\
0.93386482 0.94734264\\
0.52162234 0.92382979\\
0.89520169 0.19899726\\
};
\addlegendentry{class 1};

\addplot [
color=red,
only marks,
mark=+,
mark options={solid}
]
table[row sep=crcr]{
0.50301449 0.33508338\\
0.30620855 0.47115591\\
0.11216371 0.14931034\\
0.44328996 0.13586439\\
0.014668875 0.72578931\\
0.72406171 0.35841852\\
0.2816336 0.28527941\\
0.82886448 0.13471797\\
0.16627012 0.022493293\\
0.39390601 0.26219945\\
0.46080617 0.18033067\\
0.44530705 0.032418564\\
0.087744606 0.73392626\\
0.36629985 0.27602966\\
0.30253382 0.36845815\\
0.014233016 0.5694806\\
0.22190808 0.65861265\\
0.2280389 0.15259362\\
0.1721997 0.19186326\\
0.049046828 0.7720878\\
0.89481276 0.44158549\\
0.28614965 0.48305999\\
0.25120055 0.60810566\\
0.13098247 0.002025561\\
0.84767647 0.21322938\\
0.20927164 0.1034498\\
0.081073725 0.40751498\\
0.56204868 0.052692688\\
0.37489926 0.14997168\\
0.8677957 0.38437405\\
0.37217624 0.31105856\\
0.073690056 0.16853444\\
0.04949328 0.32272443\\
0.12192475 0.41090418\\
0.11706015 0.50552221\\
0.046636146 0.016197484\\
0.59791326 0.83685156\\
0.10158548 0.082612618\\
0.065314507 0.82071701\\
0.23429973 0.19302002\\
0.26421766 0.3087418\\
0.13094482 0.61663323\\
0.095413016 0.68514\\
0.014864337 0.51015303\\
0.017362693 0.96670263\\
0.24403153 0.13020167\\
0.21406286 0.56157937\\
0.60493711 0.90494922\\
0.18336363 0.065034354\\
0.68588997 0.65232469\\
0.0128905 0.0010733656\\
0.31040223 0.54176379\\
0.30729608 0.45133767\\
0.074321493 0.61856262\\
0.070669191 0.015520957\\
0.011930467 0.89085396\\
0.509206 0.50407844\\
0.074289609 0.56456859\\
0.27643472 0.48409788\\
0.31392999 0.47101215\\
0.94770338 0.67676529\\
0.11307983 0.73740432\\
0.15637612 0.5039276\\
0.12211877 0.62908555\\
0.27215744 0.1054805\\
0.41943272 0.14142585\\
0.21299366 0.4569683\\
0.035599967 0.78813305\\
0.081163704 0.28106398\\
0.85056678 0.22478673\\
0.46615498 0.0073288452\\
0.91376324 0.58873933\\
0.22857671 0.54211787\\
0.42219483 0.11888856\\
0.19186575 0.48558916\\
0.11121644 0.95222323\\
0.96916627 0.47865839\\
0.023743871 0.5265224\\
0.026876593 0.19300765\\
0.25067278 0.76754963\\
0.13718931 0.81330552\\
};
\addlegendentry{class -1};

\end{axis}
\end{tikzpicture}%
	\caption{}
	\label{fig:decisionBoundary2}
\end{figure}

\begin{figure}
	\centering
	\setlength\figureheight{7cm} 
	\setlength\figurewidth{9cm}
	% This file was created by matlab2tikz v0.4.4 running on MATLAB 8.0.
% Copyright (c) 2008--2013, Nico Schlömer <nico.schloemer@gmail.com>
% All rights reserved.
% 
% The latest updates can be retrieved from
%   http://www.mathworks.com/matlabcentral/fileexchange/22022-matlab2tikz
% where you can also make suggestions and rate matlab2tikz.
% 
%
% defining custom colors
\definecolor{mycolor1}{rgb}{1,0,1}%
%
\begin{tikzpicture}

\begin{axis}[%
width=\figurewidth,
height=\figureheight,
colormap/jet,
unbounded coords=jump,
scale only axis,
xmin=-0.0880227,
xmax=1.099484837,
xlabel={${\text{x}}$},
ymin=-0.0988798014,
ymax=1.090901487,
ylabel={${\text{y}}$},
title={Decision boundary for target data set 1 with 100 epochs},
legend style={draw=black,fill=white,legend cell align=left}
]

\addplot[area legend,solid,draw=mycolor1]
table[row sep=crcr]{
x y\\
-2.75866956626581 6.28318530717959 \\
-2.75047067663685 6.26802202012841 \\
-2.73138153522019 6.23271795531469 \\
-2.70409350417457 6.18225060344979 \\
-2.70000332477195 6.17468609567137 \\
-2.67680547312895 6.1317832515849 \\
-2.64953597290705 6.08135017121433 \\
-2.64951744208333 6.08131589972 \\
-2.62222941103771 6.03084854785511 \\
-2.59906862104216 5.9880142467573 \\
-2.59494137999209 5.98038119599021 \\
-2.56765334894647 5.92991384412531 \\
-2.54860126917726 5.89467832230026 \\
-2.54036531790085 5.87944649226042 \\
-2.51307728685523 5.82897914039552 \\
-2.49813391731237 5.80134239784323 \\
-2.48578925580961 5.77851178853062 \\
-2.45850122476399 5.72804443666573 \\
-2.44766656544747 5.70800647338619 \\
-2.43121319371837 5.67757708480083 \\
-2.40392516267275 5.62710973293594 \\
-2.39719921358257 5.61467054892916 \\
-2.37663713162713 5.57664238107104 \\
-2.34934910058151 5.52617502920614 \\
-2.34673186171768 5.52133462447212 \\
-2.32206106953589 5.47570767734125 \\
-2.29626450985278 5.42799870001508 \\
-2.29477303849027 5.42524032547635 \\
-2.26748500744465 5.37477297361145 \\
-2.24579715798788 5.33466277555805 \\
-2.24019697639903 5.32430562174656 \\
-2.2129089453534 5.27383826988166 \\
-2.19532980612299 5.24132685110101 \\
-2.18562091430778 5.22337091801676 \\
-2.15833288326216 5.17290356615187 \\
-2.14486245425809 5.14799092664398 \\
-2.13104485221654 5.12243621428697 \\
-2.10375682117092 5.07196886242208 \\
-2.09439510239319 5.05465500218694 \\
-2.0764687901253 5.02150151055718 \\
-2.04918075907968 4.97103415869228 \\
-2.0439277505283 4.96131907772991 \\
-2.02189272803406 4.92056680682739 \\
-1.99460469698844 4.87009945496249 \\
-1.9934603986634 4.86798315327287 \\
-1.96731666594282 4.81963210309759 \\
-1.94299304679851 4.77464722881584 \\
-1.9400286348972 4.7691647512327 \\
-1.91274060385158 4.7186973993678 \\
-1.89252569493361 4.6813113043588 \\
-1.88545257280596 4.6682300475029 \\
-1.85816454176034 4.61776269563801 \\
-1.84205834306871 4.58797537990177 \\
-1.83087651071472 4.56729534377311 \\
-1.8035884796691 4.51682799190822 \\
-1.79159099120382 4.49463945544473 \\
-1.77630044862348 4.46636064004332 \\
-1.74901241757786 4.41589328817842 \\
-1.74112363933892 4.40130353098769 \\
-1.72172438653224 4.36542593631353 \\
-1.69443635548662 4.31495858444863 \\
-1.69065628747403 4.30796760653066 \\
-1.667148324441 4.26449123258374 \\
-1.64018893560913 4.21463168207362 \\
-1.63986029339538 4.21402388071884 \\
-1.61257226234976 4.16355652885394 \\
-1.58972158374423 4.12129575761659 \\
-1.58528423130414 4.11308917698905 \\
-1.55799620025852 4.06262182512415 \\
-1.53925423187934 4.02795983315955 \\
-1.5307081692129 4.01215447325925 \\
-1.50342013816728 3.96168712139436 \\
-1.48878688001444 3.93462390870251 \\
-1.47613210712166 3.91121976952946 \\
-1.44884407607604 3.86075241766457 \\
-1.43831952814954 3.84128798424548 \\
-1.42155604503042 3.81028506579967 \\
-1.3942680139848 3.75981771393477 \\
-1.38785217628465 3.74795205978844 \\
-1.36697998293918 3.70935036206988 \\
-1.33969195189356 3.65888301020498 \\
-1.33738482441975 3.65461613533141 \\
-1.31240392084794 3.60841565834008 \\
-1.28691747255485 3.56128021087437 \\
-1.28511588980232 3.55794830647519 \\
-1.2578278587567 3.50748095461029 \\
-1.23645012068996 3.46794428641733 \\
-1.23053982771108 3.45701360274539 \\
-1.20325179666546 3.4065462508805 \\
-1.18598276882506 3.3746083619603 \\
-1.17596376561984 3.3560788990156 \\
-1.14867573457422 3.30561154715071 \\
-1.13551541696017 3.28127243750327 \\
-1.1213877035286 3.25514419528581 \\
-1.09409967248298 3.20467684342091 \\
-1.08504806509527 3.18793651304623 \\
-1.06681164143736 3.15420949155602 \\
-1.03952361039174 3.10374213969112 \\
-1.03458071323037 3.09460058858919 \\
-1.01223557934612 3.05327478782622 \\
-0.984947548300496 3.00280743596133 \\
-0.984113361365477 3.00126466413216 \\
-0.957659517254877 2.95234008409643 \\
-0.933646009500581 2.90792873967512 \\
-0.930371486209255 2.90187273223154 \\
-0.903083455163636 2.85140538036664 \\
-0.883178657635685 2.81459281521809 \\
-0.875795424118015 2.80093802850174 \\
-0.848507393072396 2.75047067663685 \\
-0.832711305770789 2.72125689076105 \\
-0.821219362026776 2.70000332477195 \\
-0.793931330981155 2.64953597290705 \\
-0.782243953905892 2.62792096630402 \\
-0.766643299935534 2.59906862104216 \\
-0.739355268889915 2.54860126917726 \\
-0.731776602040996 2.53458504184698 \\
-0.712067237844295 2.49813391731237 \\
-0.684779206798674 2.44766656544747 \\
-0.681309250176099 2.44124911738994 \\
-0.657491175753054 2.39719921358257 \\
-0.630841898311203 2.34791319293291 \\
-0.630203144707434 2.34673186171768 \\
-0.602915113661814 2.29626450985278 \\
-0.580374546446308 2.25457726847587 \\
-0.575627082616193 2.24579715798788 \\
-0.548339051570573 2.19532980612299 \\
-0.529907194581411 2.16124134401884 \\
-0.521051020524954 2.14486245425809 \\
-0.493762989479333 2.0943951023932 \\
-0.479439842716515 2.0679054195618 \\
-0.466474958433713 2.0439277505283 \\
-0.439186927388092 1.9934603986634 \\
-0.428972490851618 1.97456949510477 \\
-0.411898896342472 1.94299304679851 \\
-0.384610865296852 1.89252569493361 \\
-0.378505138986722 1.88123357064773 \\
-0.357322834251233 1.84205834306871 \\
-0.330034803205612 1.79159099120382 \\
-0.328037787121826 1.78789764619069 \\
-0.302746772159992 1.74112363933892 \\
-0.27757043525693 1.69456172173366 \\
-0.275458741114372 1.69065628747403 \\
-0.248170710068751 1.64018893560913 \\
-0.227103083392033 1.60122579727662 \\
-0.220882679023131 1.58972158374423 \\
-0.193594647977511 1.53925423187934 \\
-0.176635731527137 1.50788987281959 \\
-0.16630661693189 1.48878688001444 \\
-0.139018585886271 1.43831952814954 \\
-0.12616837966224 1.41455394836255 \\
-0.11173055484065 1.38785217628465 \\
-0.0844425237950302 1.33738482441975 \\
-0.0757010277973444 1.32121802390552 \\
-0.0571544927494096 1.28691747255485 \\
-0.0298664617037898 1.23645012068996 \\
-0.0252336759324487 1.22788209944848 \\
-0.00257843065816954 1.18598276882506 \\
0.0247096003874508 1.13551541696017 \\
0.0252336759324479 1.13454617499145 \\
0.0519976314330711 1.08504806509527 \\
0.0757010277973444 1.04121025053441 \\
0.0792856624786915 1.03458071323037 \\
0.106573693524311 0.984113361365478 \\
0.126168379662241 0.947874326077373 \\
0.133861724569931 0.933646009500581 \\
0.161149755615551 0.883178657635685 \\
0.176635731527137 0.854538401620339 \\
0.188437786661173 0.832711305770788 \\
0.215725817706792 0.782243953905892 \\
0.227103083392033 0.761202477163302 \\
0.243013848752413 0.731776602040996 \\
0.270301879798033 0.6813092501761 \\
0.277570435256929 0.667866552706268 \\
0.297589910843652 0.630841898311203 \\
0.324877941889273 0.580374546446308 \\
0.328037787121827 0.57453062824923 \\
0.352165972934893 0.52990719458141 \\
0.378505138986722 0.481194703792195 \\
0.379454003980513 0.479439842716515 \\
0.406742035026133 0.428972490851619 \\
0.428972490851619 0.387858779335159 \\
0.434030066071753 0.378505138986722 \\
0.461318097117373 0.328037787121827 \\
0.479439842716515 0.294522854878124 \\
0.488606128162994 0.277570435256929 \\
0.515894159208614 0.227103083392033 \\
0.52990719458141 0.20118693042109 \\
0.543182190254234 0.176635731527137 \\
0.570470221299854 0.126168379662241 \\
0.580374546446308 0.107851005964052 \\
0.597758252345474 0.0757010277973444 \\
0.625046283391095 0.0252336759324479 \\
0.630841898311203 0.0145150815070176 \\
0.652334314436716 -0.0252336759324487 \\
0.679622345482336 -0.0757010277973444 \\
0.6813092501761 -0.0788208429500183 \\
0.706910376527956 -0.12616837966224 \\
0.731776602040996 -0.172156767407053 \\
0.734198407573576 -0.176635731527137 \\
0.761486438619196 -0.227103083392033 \\
0.782243953905892 -0.26549269186409 \\
0.788774469664816 -0.27757043525693 \\
0.816062500710436 -0.328037787121826 \\
0.832711305770788 -0.358828616321124 \\
0.843350531756056 -0.378505138986722 \\
0.870638562801676 -0.428972490851618 \\
0.883178657635685 -0.452164540778161 \\
0.897926593847297 -0.479439842716515 \\
0.925214624892917 -0.529907194581411 \\
0.933646009500581 -0.545500465235196 \\
0.952502655938537 -0.580374546446308 \\
0.979790686984158 -0.630841898311203 \\
0.984113361365478 -0.638836389692233 \\
1.00707871802978 -0.681309250176099 \\
1.0343667490754 -0.731776602040996 \\
1.03458071323037 -0.732172314149267 \\
1.06165478012102 -0.782243953905892 \\
1.08504806509527 -0.825508238606302 \\
1.08894281116664 -0.832711305770789 \\
1.11623084221226 -0.883178657635685 \\
1.13551541696017 -0.91884416306334 \\
1.14351887325788 -0.933646009500581 \\
1.1708069043035 -0.984113361365477 \\
1.18598276882506 -1.01218008752037 \\
1.19809493534912 -1.03458071323037 \\
1.22538296639474 -1.08504806509527 \\
1.23645012068996 -1.10551601197741 \\
1.25267099744036 -1.13551541696017 \\
1.27995902848598 -1.18598276882506 \\
1.28691747255485 -1.19885193643445 \\
1.3072470595316 -1.23645012068996 \\
1.33453509057722 -1.28691747255485 \\
1.33738482441975 -1.29218786089148 \\
1.36182312162284 -1.33738482441975 \\
1.38785217628465 -1.38552378534852 \\
1.38911115266846 -1.38785217628465 \\
1.41639918371408 -1.43831952814954 \\
1.43831952814954 -1.47885970980555 \\
1.4436872147597 -1.48878688001444 \\
1.47097524580532 -1.53925423187934 \\
1.48878688001444 -1.57219563426259 \\
1.49826327685094 -1.58972158374423 \\
1.52555130789656 -1.64018893560913 \\
1.53925423187934 -1.66553155871962 \\
1.55283933894218 -1.69065628747403 \\
1.5801273699878 -1.74112363933892 \\
1.58972158374423 -1.75886748317666 \\
1.60741540103342 -1.79159099120382 \\
1.63470343207904 -1.84205834306871 \\
1.64018893560913 -1.85220340763369 \\
1.66199146312466 -1.89252569493361 \\
1.68927949417028 -1.94299304679851 \\
1.69065628747403 -1.94553933209073 \\
1.7165675252159 -1.9934603986634 \\
1.74112363933892 -2.03887525654777 \\
1.74385555626152 -2.0439277505283 \\
1.77114358730714 -2.09439510239319 \\
1.79159099120382 -2.1322111810048 \\
1.79843161835276 -2.14486245425809 \\
1.82571964939838 -2.19532980612299 \\
1.84205834306871 -2.22554710546184 \\
1.853007680444 -2.24579715798788 \\
1.88029571148962 -2.29626450985278 \\
1.89252569493361 -2.31888302991887 \\
1.90758374253524 -2.34673186171768 \\
1.93487177358086 -2.39719921358257 \\
1.94299304679851 -2.41221895437591 \\
1.96215980462649 -2.44766656544747 \\
1.98944783567211 -2.49813391731237 \\
1.9934603986634 -2.50555487883294 \\
2.01673586671772 -2.54860126917726 \\
2.0439277505283 -2.59889080328998 \\
2.04402389776335 -2.59906862104216 \\
2.07131192880896 -2.64953597290705 \\
2.0943951023932 -2.69222672774702 \\
2.09859995985459 -2.70000332477195 \\
2.12588799090021 -2.75047067663685 \\
2.14486245425809 -2.78556265220405 \\
2.15317602194582 -2.80093802850174 \\
2.18046405299145 -2.85140538036664 \\
2.19532980612299 -2.87889857666109 \\
2.20775208403707 -2.90187273223154 \\
2.23504011508269 -2.95234008409643 \\
2.24579715798788 -2.97223450111812 \\
2.26232814612831 -3.00280743596133 \\
2.28961617717393 -3.05327478782622 \\
2.29626450985278 -3.06557042557516 \\
2.31690420821955 -3.10374213969112 \\
2.34419223926517 -3.15420949155602 \\
2.34673186171768 -3.15890635003219 \\
2.37148027031079 -3.20467684342091 \\
2.39719921358257 -3.25224227448923 \\
2.39876830135641 -3.25514419528581 \\
2.42605633240203 -3.30561154715071 \\
2.44766656544747 -3.34557819894626 \\
2.45334436344765 -3.3560788990156 \\
2.48063239449327 -3.4065462508805 \\
2.49813391731237 -3.4389141234033 \\
2.50792042553889 -3.45701360274539 \\
2.53520845658451 -3.50748095461029 \\
2.54860126917726 -3.53225004786034 \\
2.56249648763013 -3.55794830647519 \\
2.58978451867575 -3.60841565834008 \\
2.59906862104216 -3.62558597231737 \\
2.61707254972137 -3.65888301020498 \\
2.64436058076699 -3.70935036206988 \\
2.64953597290705 -3.71892189677441 \\
2.67164861181261 -3.75981771393477 \\
2.69893664285823 -3.81028506579967 \\
2.70000332477195 -3.81225782123145 \\
2.72622467390385 -3.86075241766456 \\
2.75047067663685 -3.90559374568848 \\
2.75351270494947 -3.91121976952946 \\
2.78080073599509 -3.96168712139436 \\
2.80093802850174 -3.99892967014552 \\
2.80808876704071 -4.01215447325925 \\
2.83537679808633 -4.06262182512415 \\
2.85140538036664 -4.09226559460255 \\
2.86266482913195 -4.11308917698905 \\
2.88995286017757 -4.16355652885394 \\
2.90187273223154 -4.18560151905959 \\
2.91724089122319 -4.21402388071884 \\
2.94452892226881 -4.26449123258374 \\
2.95234008409643 -4.27893744351662 \\
2.97181695331443 -4.31495858444863 \\
2.99910498436005 -4.36542593631353 \\
3.00280743596133 -4.37227336797366 \\
3.02639301540567 -4.41589328817842 \\
3.05327478782622 -4.46560929243069 \\
3.05368104645129 -4.46636064004332 \\
3.08096907749691 -4.51682799190822 \\
3.10374213969112 -4.55894521688773 \\
3.10825710854253 -4.56729534377311 \\
3.13554513958815 -4.61776269563801 \\
3.15420949155602 -4.65228114134477 \\
3.16283317063377 -4.66823004750291 \\
3.19012120167939 -4.7186973993678 \\
3.20467684342091 -4.7456170658018 \\
3.21740923272501 -4.7691647512327 \\
3.24469726377063 -4.81963210309759 \\
3.25514419528581 -4.83895299025884 \\
3.27198529481625 -4.87009945496249 \\
3.29927332586187 -4.92056680682739 \\
3.30561154715071 -4.93228891471587 \\
3.32656135690749 -4.97103415869228 \\
3.35384938795311 -5.02150151055718 \\
3.3560788990156 -5.02562483917291 \\
3.38113741899873 -5.07196886242208 \\
3.4065462508805 -5.11896076362994 \\
3.40842545004436 -5.12243621428697 \\
3.43571348108997 -5.17290356615187 \\
3.45701360274539 -5.21229668808698 \\
3.46300151213559 -5.22337091801676 \\
3.49028954318121 -5.27383826988166 \\
3.50748095461029 -5.30563261254401 \\
3.51757757422684 -5.32430562174656 \\
3.54486560527246 -5.37477297361145 \\
3.55794830647519 -5.39896853700105 \\
3.57215363631807 -5.42524032547635 \\
3.5994416673637 -5.47570767734125 \\
3.60841565834008 -5.49230446145809 \\
3.62672969840932 -5.52617502920614 \\
3.65401772945494 -5.57664238107104 \\
3.65888301020498 -5.58564038591512 \\
3.68130576050056 -5.62710973293594 \\
3.70859379154618 -5.67757708480083 \\
3.70935036206988 -5.67897631037216 \\
3.7358818225918 -5.72804443666573 \\
3.75981771393477 -5.77231223482919 \\
3.76316985363742 -5.77851178853062 \\
3.79045788468304 -5.82897914039552 \\
3.81028506579967 -5.86564815928623 \\
3.81774591572866 -5.87944649226042 \\
3.84503394677428 -5.92991384412531 \\
3.86075241766457 -5.95898408374327 \\
3.8723219778199 -5.98038119599021 \\
3.89961000886552 -6.03084854785511 \\
3.91121976952946 -6.0523200082003 \\
3.92689803991114 -6.08131589972 \\
3.95418607095676 -6.1317832515849 \\
3.96168712139436 -6.14565593265733 \\
3.98147410200238 -6.18225060344979 \\
4.008762133048 -6.23271795531469 \\
4.01215447325925 -6.23899185711437 \\
4.03605016409362 -6.28318530717959 \\
NaN NaN \\
};

\addlegendentry{decision boundary};

\addplot [
color=green,
only marks,
mark=asterisk,
mark options={solid}
]
table[row sep=crcr]{
0.39412761 0.94238725\\
0.50301449 0.33508338\\
0.72197985 0.43736382\\
0.46676255 0.53249823\\
0.66405187 0.39870305\\
0.72406171 0.35841852\\
0.26181868 0.86863524\\
0.7084714 0.62641267\\
0.78385902 0.24117231\\
0.98615781 0.97808165\\
0.47334271 0.64050078\\
0.90281883 0.22984865\\
0.45105876 0.68133515\\
0.80451681 0.66582341\\
0.82886448 0.13471797\\
0.71812397 0.069318238\\
0.56918952 0.8529305\\
0.44348322 0.53651682\\
0.8518447 0.012886306\\
0.75947939 0.88920608\\
0.94975928 0.8660206\\
0.55793851 0.25424693\\
0.59617708 0.15926482\\
0.81620571 0.59436442\\
0.97709235 0.33110008\\
0.70368367 0.86363422\\
0.52206092 0.56762331\\
0.93289706 0.98048127\\
0.71335444 0.7918316\\
0.4496421 0.83302723\\
0.96882014 0.63898657\\
0.3557161 0.66900008\\
0.75533857 0.37981775\\
0.89481276 0.44158549\\
0.93273619 0.17599564\\
0.94081954 0.790224\\
0.70185317 0.5136085\\
0.84767647 0.21322938\\
0.8511224 0.40775702\\
0.3192963 0.94181515\\
0.8677957 0.38437405\\
0.199838 0.89664815\\
0.56670978 0.73399632\\
0.52211176 0.39979385\\
0.76991848 0.16930586\\
0.3750558 0.52474549\\
0.8233872 0.64120266\\
0.59791326 0.83685156\\
0.94915039 0.80346219\\
0.28879765 0.69778483\\
0.88883261 0.46188778\\
0.93309842 0.44535473\\
0.99953167 0.87535098\\
0.21198796 0.8352594\\
0.4984098 0.33309507\\
0.2904883 0.88070533\\
0.67275441 0.47968681\\
0.95799112 0.56081673\\
0.76655153 0.61590866\\
0.66612374 0.66189901\\
0.28819335 0.71396084\\
0.81673121 0.5152079\\
0.98548351 0.60586579\\
0.81939293 0.8221174\\
0.62113871 0.31775068\\
0.56022204 0.58769677\\
0.82200759 0.25435426\\
0.26321193 0.8030309\\
0.75363453 0.66784589\\
0.6596448 0.013626353\\
0.60211691 0.45456094\\
0.60493711 0.90494922\\
0.65950155 0.28215885\\
0.63654691 0.47659189\\
0.17030911 0.98371192\\
0.53960111 0.92234958\\
0.62339418 0.56119625\\
0.68588997 0.65232469\\
0.67734595 0.7726796\\
0.87683002 0.10617579\\
0.77907925 0.0068577917\\
0.92667831 0.19566205\\
0.67871988 0.78714313\\
0.22715372 0.76170319\\
0.516252 0.90703504\\
0.45820422 0.7585686\\
0.70320334 0.38072994\\
0.58248388 0.33111079\\
0.509206 0.50407844\\
0.37960413 0.77986778\\
0.77088184 0.80221322\\
0.63819322 0.2027589\\
0.98656686 0.57961489\\
0.5028758 0.66650026\\
0.94770338 0.67676529\\
0.82802584 0.94251124\\
0.9175572 0.77014853\\
0.81212591 0.86626227\\
0.90826256 0.99094832\\
0.76266647 0.79261032\\
0.72180028 0.44864921\\
0.65163982 0.52435711\\
0.75402267 0.17147322\\
0.66316201 0.13066575\\
0.8834935 0.21878112\\
0.85056678 0.22478673\\
0.34020345 0.90887473\\
0.91376324 0.58873933\\
0.8620449 0.65352398\\
0.65661854 0.31343498\\
0.8911828 0.23115882\\
0.48814352 0.41606383\\
0.99264565 0.29879879\\
0.37332594 0.67243639\\
0.53137832 0.93825748\\
0.50194434 0.56296257\\
0.6604274 0.16902126\\
0.67365301 0.27889542\\
0.95733018 0.55681466\\
0.56505387 0.23192084\\
0.96916627 0.47865839\\
0.87021582 0.79272079\\
0.51952869 0.90960045\\
0.19229142 0.92219639\\
0.71568905 0.013266366\\
0.25067278 0.76754963\\
0.93386482 0.94734264\\
0.52162234 0.92382979\\
0.89520169 0.19899726\\
};
\addlegendentry{class 1};

\addplot [
color=red,
only marks,
mark=+,
mark options={solid}
]
table[row sep=crcr]{
0.30620855 0.47115591\\
0.11216371 0.14931034\\
0.44328996 0.13586439\\
0.014668875 0.72578931\\
0.2816336 0.28527941\\
0.16627012 0.022493293\\
0.39390601 0.26219945\\
0.52075748 0.11651516\\
0.46080617 0.18033067\\
0.44530705 0.032418564\\
0.087744606 0.73392626\\
0.36629985 0.27602966\\
0.30253382 0.36845815\\
0.014233016 0.5694806\\
0.22190808 0.65861265\\
0.2280389 0.15259362\\
0.1721997 0.19186326\\
0.049046828 0.7720878\\
0.28614965 0.48305999\\
0.25120055 0.60810566\\
0.13098247 0.002025561\\
0.20927164 0.1034498\\
0.45509169 0.15733667\\
0.081073725 0.40751498\\
0.56204868 0.052692688\\
0.37489926 0.14997168\\
0.37217624 0.31105856\\
0.073690056 0.16853444\\
0.04949328 0.32272443\\
0.12192475 0.41090418\\
0.11706015 0.50552221\\
0.046636146 0.016197484\\
0.10158548 0.082612618\\
0.065314507 0.82071701\\
0.23429973 0.19302002\\
0.063127896 0.01295783\\
0.26421766 0.3087418\\
0.13094482 0.61663323\\
0.095413016 0.68514\\
0.014864337 0.51015303\\
0.017362693 0.96670263\\
0.24403153 0.13020167\\
0.21406286 0.56157937\\
0.18336363 0.065034354\\
0.0128905 0.0010733656\\
0.31040223 0.54176379\\
0.30729608 0.45133767\\
0.074321493 0.61856262\\
0.070669191 0.015520957\\
0.011930467 0.89085396\\
0.074289609 0.56456859\\
0.19323631 0.76719736\\
0.27643472 0.48409788\\
0.31392999 0.47101215\\
0.11307983 0.73740432\\
0.15637612 0.5039276\\
0.12211877 0.62908555\\
0.27215744 0.1054805\\
0.41943272 0.14142585\\
0.21299366 0.4569683\\
0.035599967 0.78813305\\
0.081163704 0.28106398\\
0.46615498 0.0073288452\\
0.22857671 0.54211787\\
0.18131633 0.34314769\\
0.42219483 0.11888856\\
0.19186575 0.48558916\\
0.11121644 0.95222323\\
0.023743871 0.5265224\\
0.026876593 0.19300765\\
0.13718931 0.81330552\\
};
\addlegendentry{class -1};

\end{axis}
\end{tikzpicture}%
	\caption{}
	\label{fig:decisionBoundary1100}
\end{figure}

\begin{figure}
	\centering
	\setlength\figureheight{7cm} 
	\setlength\figurewidth{9cm}
	% This file was created by matlab2tikz v0.4.4 running on MATLAB 8.0.
% Copyright (c) 2008--2013, Nico Schlömer <nico.schloemer@gmail.com>
% All rights reserved.
% 
% The latest updates can be retrieved from
%   http://www.mathworks.com/matlabcentral/fileexchange/22022-matlab2tikz
% where you can also make suggestions and rate matlab2tikz.
% 
%
% defining custom colors
\definecolor{mycolor1}{rgb}{1,0,1}%
%
\begin{tikzpicture}

\begin{axis}[%
width=\figurewidth,
height=\figureheight,
colormap/jet,
unbounded coords=jump,
scale only axis,
xmin=-0.0880227,
xmax=1.099484837,
xlabel={${\text{x}}$},
ymin=-0.0988798014,
ymax=1.090901487,
ylabel={${\text{y}}$},
title={Decision boundary for target data set 2 with 100 epochs},
legend style={draw=black,fill=white,legend cell align=left}
]

\addplot[area legend,solid,draw=mycolor1]
table[row sep=crcr]{
x y\\
-2.72809116159936 6.28318530717959 \\
-2.70067834859984 6.23271795531469 \\
-2.70000332477195 6.23147522750561 \\
-2.67326553560033 6.18225060344979 \\
-2.64953597290705 6.13856416419007 \\
-2.64585272260081 6.1317832515849 \\
-2.61843990960129 6.08131589972 \\
-2.59906862104216 6.04565310087453 \\
-2.59102709660177 6.03084854785511 \\
-2.56361428360226 5.98038119599021 \\
-2.54860126917726 5.95274203755899 \\
-2.53620147060274 5.92991384412531 \\
-2.50878865760322 5.87944649226042 \\
-2.49813391731237 5.85983097424345 \\
-2.4813758446037 5.82897914039552 \\
-2.45396303160419 5.77851178853062 \\
-2.44766656544747 5.76691991092791 \\
-2.42655021860467 5.72804443666573 \\
-2.39913740560515 5.67757708480083 \\
-2.39719921358257 5.67400884761237 \\
-2.37172459260563 5.62710973293594 \\
-2.34673186171768 5.58109778429683 \\
-2.34431177960612 5.57664238107104 \\
-2.3168989666066 5.52617502920614 \\
-2.29626450985278 5.48818672098129 \\
-2.28948615360708 5.47570767734125 \\
-2.26207334060756 5.42524032547635 \\
-2.24579715798788 5.39527565766575 \\
-2.23466052760805 5.37477297361145 \\
-2.20724771460853 5.32430562174656 \\
-2.19532980612299 5.30236459435021 \\
-2.17983490160901 5.27383826988166 \\
-2.15242208860949 5.22337091801676 \\
-2.14486245425809 5.20945353103467 \\
-2.12500927560998 5.17290356615187 \\
-2.09759646261046 5.12243621428697 \\
-2.09439510239319 5.11654246771912 \\
-2.07018364961094 5.07196886242208 \\
-2.0439277505283 5.02363140440358 \\
-2.04277083661143 5.02150151055718 \\
-2.01535802361191 4.97103415869228 \\
-1.9934603986634 4.93072034108804 \\
-1.98794521061239 4.92056680682739 \\
-1.96053239761287 4.87009945496249 \\
-1.94299304679851 4.8378092777725 \\
-1.93311958461336 4.81963210309759 \\
-1.90570677161384 4.7691647512327 \\
-1.89252569493361 4.74489821445696 \\
-1.87829395861432 4.7186973993678 \\
-1.8508811456148 4.6682300475029 \\
-1.84205834306871 4.65198715114142 \\
-1.82346833261528 4.61776269563801 \\
-1.79605551961577 4.56729534377311 \\
-1.79159099120382 4.55907608782588 \\
-1.76864270661625 4.51682799190822 \\
-1.74122989361673 4.46636064004332 \\
-1.74112363933892 4.46616502451034 \\
-1.71381708061722 4.41589328817842 \\
-1.69065628747403 4.3732539611948 \\
-1.6864042676177 4.36542593631353 \\
-1.65899145461818 4.31495858444863 \\
-1.64018893560913 4.28034289787926 \\
-1.63157864161866 4.26449123258374 \\
-1.60416582861915 4.21402388071884 \\
-1.58972158374423 4.18743183456372 \\
-1.57675301561963 4.16355652885394 \\
-1.54934020262011 4.11308917698905 \\
-1.53925423187934 4.09452077124817 \\
-1.52192738962059 4.06262182512415 \\
-1.49451457662108 4.01215447325925 \\
-1.48878688001444 4.00160970793263 \\
-1.46710176362156 3.96168712139436 \\
-1.43968895062204 3.91121976952946 \\
-1.43831952814954 3.90869864461709 \\
-1.41227613762252 3.86075241766457 \\
-1.38785217628465 3.81578758130155 \\
-1.38486332462301 3.81028506579967 \\
-1.35745051162349 3.75981771393477 \\
-1.33738482441975 3.72287651798601 \\
-1.33003769862397 3.70935036206988 \\
-1.30262488562445 3.65888301020498 \\
-1.28691747255485 3.62996545467047 \\
-1.27521207262494 3.60841565834008 \\
-1.24779925962542 3.55794830647519 \\
-1.23645012068996 3.53705439135493 \\
-1.2203864466259 3.50748095461029 \\
-1.19297363362638 3.45701360274539 \\
-1.18598276882506 3.44414332803939 \\
-1.16556082062687 3.4065462508805 \\
-1.13814800762735 3.3560788990156 \\
-1.13551541696017 3.35123226472385 \\
-1.11073519462783 3.30561154715071 \\
-1.08504806509527 3.2583212014083 \\
-1.08332238162832 3.25514419528581 \\
-1.0559095686288 3.20467684342091 \\
-1.03458071323037 3.16541013809276 \\
-1.02849675562928 3.15420949155602 \\
-1.00108394262976 3.10374213969112 \\
-0.984113361365477 3.07249907477722 \\
-0.973671129630244 3.05327478782622 \\
-0.946258316630727 3.00280743596133 \\
-0.933646009500581 2.97958801146168 \\
-0.91884550363121 2.95234008409643 \\
-0.891432690631692 2.90187273223154 \\
-0.883178657635685 2.88667694814614 \\
-0.864019877632175 2.85140538036664 \\
-0.836607064632658 2.80093802850174 \\
-0.832711305770789 2.7937658848306 \\
-0.809194251633141 2.75047067663685 \\
-0.782243953905892 2.70085482151506 \\
-0.781781438633624 2.70000332477195 \\
-0.754368625634106 2.64953597290705 \\
-0.731776602040996 2.60794375819952 \\
-0.726955812634588 2.59906862104216 \\
-0.69954299963507 2.54860126917726 \\
-0.681309250176099 2.51503269488398 \\
-0.672130186635553 2.49813391731237 \\
-0.644717373636035 2.44766656544747 \\
-0.630841898311203 2.42212163156843 \\
-0.617304560636518 2.39719921358257 \\
-0.589891747637 2.34673186171768 \\
-0.580374546446308 2.3292105682529 \\
-0.562478934637484 2.29626450985278 \\
-0.535066121637966 2.24579715798788 \\
-0.529907194581411 2.23629950493735 \\
-0.507653308638449 2.19532980612299 \\
-0.480240495638931 2.14486245425809 \\
-0.479439842716515 2.14338844162181 \\
-0.452827682639414 2.0943951023932 \\
-0.428972490851618 2.05047737830627 \\
-0.425414869639896 2.0439277505283 \\
-0.398002056640378 1.9934603986634 \\
-0.378505138986722 1.95756631499073 \\
-0.370589243640861 1.94299304679851 \\
-0.343176430641344 1.89252569493361 \\
-0.328037787121826 1.86465525167519 \\
-0.315763617641827 1.84205834306871 \\
-0.288350804642309 1.79159099120382 \\
-0.27757043525693 1.77174418835965 \\
-0.260937991642791 1.74112363933892 \\
-0.233525178643274 1.69065628747403 \\
-0.227103083392033 1.67883312504411 \\
-0.206112365643756 1.64018893560913 \\
-0.178699552644239 1.58972158374423 \\
-0.176635731527137 1.58592206172857 \\
-0.151286739644722 1.53925423187934 \\
-0.12616837966224 1.49301099841303 \\
-0.123873926645205 1.48878688001444 \\
-0.0964611136456871 1.43831952814954 \\
-0.0757010277973444 1.40009993509749 \\
-0.0690483006461688 1.38785217628465 \\
-0.041635487646652 1.33738482441975 \\
-0.0252336759324487 1.30718887178194 \\
-0.0142226746471342 1.28691747255485 \\
0.0131901383523824 1.23645012068996 \\
0.0252336759324479 1.2142778084664 \\
0.0406029513519005 1.18598276882506 \\
0.0680157643514178 1.13551541696017 \\
0.0757010277973444 1.12136674515086 \\
0.0954285773509359 1.08504806509527 \\
0.122841390350453 1.03458071323037 \\
0.126168379662241 1.02845568183532 \\
0.15025420334997 0.984113361365478 \\
0.176635731527137 0.93554461851978 \\
0.177667016349487 0.933646009500581 \\
0.205079829349004 0.883178657635685 \\
0.227103083392033 0.842633555204239 \\
0.232492642348523 0.832711305770788 \\
0.25990545534804 0.782243953905892 \\
0.277570435256929 0.749722491888698 \\
0.287318268347557 0.731776602040996 \\
0.314731081347075 0.6813092501761 \\
0.328037787121827 0.656811428573155 \\
0.342143894346592 0.630841898311203 \\
0.369556707346109 0.580374546446308 \\
0.378505138986722 0.563900365257616 \\
0.396969520345628 0.52990719458141 \\
0.424382333345145 0.479439842716515 \\
0.428972490851619 0.470989301942074 \\
0.451795146344662 0.428972490851619 \\
0.479207959344179 0.378505138986722 \\
0.479439842716515 0.378078238626534 \\
0.506620772343696 0.328037787121827 \\
0.52990719458141 0.285167175310994 \\
0.534033585343214 0.277570435256929 \\
0.561446398342732 0.227103083392033 \\
0.580374546446308 0.192256111995451 \\
0.58885921134225 0.176635731527137 \\
0.616272024341766 0.126168379662241 \\
0.630841898311203 0.099345048679911 \\
0.643684837341283 0.0757010277973444 \\
0.671097650340801 0.0252336759324479 \\
0.6813092501761 0.00643398536436908 \\
0.698510463340319 -0.0252336759324487 \\
0.725923276339836 -0.0757010277973444 \\
0.731776602040996 -0.0864770779511703 \\
0.753336089339353 -0.12616837966224 \\
0.780748902338871 -0.176635731527137 \\
0.782243953905892 -0.179388141266712 \\
0.808161715338389 -0.227103083392033 \\
0.832711305770788 -0.272299204582252 \\
0.835574528337906 -0.27757043525693 \\
0.862987341337424 -0.328037787121826 \\
0.883178657635685 -0.365210267897795 \\
0.890400154336941 -0.378505138986722 \\
0.917812967336458 -0.428972490851618 \\
0.933646009500581 -0.458121331213335 \\
0.945225780335975 -0.479439842716515 \\
0.972638593335493 -0.529907194581411 \\
0.984113361365478 -0.551032394528877 \\
1.00005140633501 -0.580374546446308 \\
1.02746421933453 -0.630841898311203 \\
1.03458071323037 -0.643943457844417 \\
1.05487703233404 -0.681309250176099 \\
1.08228984533356 -0.731776602040996 \\
1.08504806509527 -0.736854521159957 \\
1.10970265833308 -0.782243953905892 \\
1.13551541696017 -0.8297655844755 \\
1.1371154713326 -0.832711305770789 \\
1.16452828433212 -0.883178657635685 \\
1.18598276882506 -0.92267664779104 \\
1.19194109733163 -0.933646009500581 \\
1.21935391033115 -0.984113361365477 \\
1.23645012068996 -1.01558771110658 \\
1.24676672333067 -1.03458071323037 \\
1.27417953633019 -1.08504806509527 \\
1.28691747255485 -1.10849877442212 \\
1.3015923493297 -1.13551541696017 \\
1.32900516232922 -1.18598276882506 \\
1.33738482441975 -1.20140983773766 \\
1.35641797532874 -1.23645012068996 \\
1.38383078832826 -1.28691747255485 \\
1.38785217628465 -1.2943209010532 \\
1.41124360132777 -1.33738482441975 \\
1.43831952814954 -1.38723196436875 \\
1.43865641432729 -1.38785217628465 \\
1.46606922732681 -1.43831952814954 \\
1.48878688001444 -1.48014302768429 \\
1.49348204032632 -1.48878688001444 \\
1.52089485332584 -1.53925423187934 \\
1.53925423187934 -1.57305409099983 \\
1.54830766632536 -1.58972158374423 \\
1.57572047932488 -1.64018893560913 \\
1.58972158374423 -1.66596515431537 \\
1.60313329232439 -1.69065628747403 \\
1.63054610532391 -1.74112363933892 \\
1.64018893560913 -1.75887621763091 \\
1.65795891832343 -1.79159099120382 \\
1.68537173132295 -1.84205834306871 \\
1.69065628747403 -1.85178728094645 \\
1.71278454432246 -1.89252569493361 \\
1.74019735732198 -1.94299304679851 \\
1.74112363933892 -1.94469834426199 \\
1.7676101703215 -1.9934603986634 \\
1.79159099120382 -2.03760940757753 \\
1.79502298332102 -2.0439277505283 \\
1.82243579632053 -2.09439510239319 \\
1.84205834306871 -2.13052047089307 \\
1.84984860932005 -2.14486245425809 \\
1.87726142231957 -2.19532980612299 \\
1.89252569493361 -2.22343153420861 \\
1.90467423531909 -2.24579715798788 \\
1.9320870483186 -2.29626450985278 \\
1.94299304679851 -2.31634259752415 \\
1.95949986131812 -2.34673186171768 \\
1.98691267431764 -2.39719921358257 \\
1.9934603986634 -2.40925366083969 \\
2.01432548731716 -2.44766656544747 \\
2.04173830031667 -2.49813391731237 \\
2.0439277505283 -2.50216472415524 \\
2.06915111331619 -2.54860126917726 \\
2.0943951023932 -2.59507578747078 \\
2.09656392631571 -2.59906862104216 \\
2.12397673931523 -2.64953597290705 \\
2.14486245425809 -2.68798685078632 \\
2.15138955231474 -2.70000332477195 \\
2.17880236531426 -2.75047067663685 \\
2.19532980612299 -2.78089791410186 \\
2.20621517831378 -2.80093802850174 \\
2.23362799131329 -2.85140538036664 \\
2.24579715798788 -2.8738089774174 \\
2.26104080431281 -2.90187273223154 \\
2.28845361731233 -2.95234008409643 \\
2.29626450985278 -2.96672004073294 \\
2.31586643031185 -3.00280743596133 \\
2.34327924331137 -3.05327478782622 \\
2.34673186171768 -3.05963110404848 \\
2.37069205631088 -3.10374213969112 \\
2.39719921358257 -3.15254216736402 \\
2.3981048693104 -3.15420949155602 \\
2.42551768230992 -3.20467684342091 \\
2.44766656544747 -3.24545323067956 \\
2.45293049530943 -3.25514419528581 \\
2.48034330830895 -3.30561154715071 \\
2.49813391731237 -3.33836429399511 \\
2.50775612130847 -3.3560788990156 \\
2.53516893430799 -3.4065462508805 \\
2.54860126917726 -3.43127535731065 \\
2.56258174730751 -3.45701360274539 \\
2.58999456030702 -3.50748095461029 \\
2.59906862104216 -3.52418642062619 \\
2.61740737330654 -3.55794830647519 \\
2.64482018630606 -3.60841565834008 \\
2.64953597290705 -3.61709748394173 \\
2.67223299930557 -3.65888301020498 \\
2.69964581230509 -3.70935036206988 \\
2.70000332477195 -3.71000854725727 \\
2.72705862530461 -3.75981771393477 \\
2.75047067663685 -3.80291961057281 \\
2.75447143830413 -3.81028506579967 \\
2.78188425130364 -3.86075241766456 \\
2.80093802850174 -3.89583067388835 \\
2.80929706430316 -3.91121976952946 \\
2.83670987730268 -3.96168712139436 \\
2.85140538036664 -3.98874173720389 \\
2.8641226903022 -4.01215447325925 \\
2.89153550330171 -4.06262182512415 \\
2.90187273223154 -4.08165280051943 \\
2.91894831630123 -4.11308917698905 \\
2.94636112930075 -4.16355652885394 \\
2.95234008409643 -4.17456386383497 \\
2.97377394230027 -4.21402388071884 \\
3.00118675529978 -4.26449123258374 \\
3.00280743596133 -4.26747492715052 \\
3.0285995682993 -4.31495858444863 \\
3.05327478782622 -4.36038599046605 \\
3.05601238129882 -4.36542593631353 \\
3.08342519429833 -4.41589328817842 \\
3.10374213969112 -4.4532970537816 \\
3.11083800729785 -4.46636064004332 \\
3.13825082029737 -4.51682799190822 \\
3.15420949155602 -4.54620811709714 \\
3.16566363329689 -4.56729534377311 \\
3.19307644629641 -4.61776269563801 \\
3.20467684342091 -4.63911918041268 \\
3.22048925929592 -4.66823004750291 \\
3.24790207229544 -4.7186973993678 \\
3.25514419528581 -4.73203024372822 \\
3.27531488529496 -4.7691647512327 \\
3.30272769829447 -4.81963210309759 \\
3.30561154715071 -4.82494130704376 \\
3.33014051129399 -4.87009945496249 \\
3.3560788990156 -4.9178523703593 \\
3.35755332429351 -4.92056680682739 \\
3.38496613729303 -4.97103415869228 \\
3.4065462508805 -5.01076343367484 \\
3.41237895029255 -5.02150151055718 \\
3.43979176329206 -5.07196886242208 \\
3.45701360274539 -5.10367449699039 \\
3.46720457629158 -5.12243621428697 \\
3.4946173892911 -5.17290356615187 \\
3.50748095461029 -5.19658556030592 \\
3.52203020229061 -5.22337091801676 \\
3.54944301529013 -5.27383826988166 \\
3.55794830647519 -5.28949662362146 \\
3.57685582828965 -5.32430562174656 \\
3.60426864128917 -5.37477297361145 \\
3.60841565834008 -5.38240768693701 \\
3.63168145428868 -5.42524032547635 \\
3.65888301020498 -5.47531875025255 \\
3.6590942672882 -5.47570767734125 \\
3.68650708028772 -5.52617502920614 \\
3.70935036206988 -5.56822981356809 \\
3.71391989328724 -5.57664238107104 \\
3.74133270628675 -5.62710973293594 \\
3.75981771393477 -5.66114087688363 \\
3.76874551928627 -5.67757708480083 \\
3.79615833228579 -5.72804443666573 \\
3.81028506579967 -5.75405194019917 \\
3.82357114528531 -5.77851178853062 \\
3.85098395828482 -5.82897914039552 \\
3.86075241766457 -5.84696300351471 \\
3.87839677128434 -5.87944649226042 \\
3.90580958428386 -5.92991384412531 \\
3.91121976952946 -5.93987406683025 \\
3.93322239728338 -5.98038119599021 \\
3.96063521028289 -6.03084854785511 \\
3.96168712139436 -6.03278513014579 \\
3.98804802328241 -6.08131589972 \\
4.01215447325925 -6.12569619346133 \\
4.01546083628193 -6.1317832515849 \\
4.04287364928144 -6.18225060344979 \\
4.06262182512415 -6.21860725677688 \\
4.07028646228096 -6.23271795531469 \\
4.09769927528048 -6.28318530717959 \\
NaN NaN \\
};

\addlegendentry{decision boundary};

\addplot [
color=green,
only marks,
mark=asterisk,
mark options={solid}
]
table[row sep=crcr]{
0.39412761 0.94238725\\
0.72197985 0.43736382\\
0.46676255 0.53249823\\
0.66405187 0.39870305\\
0.26181868 0.86863524\\
0.7084714 0.62641267\\
0.78385902 0.24117231\\
0.98615781 0.97808165\\
0.47334271 0.64050078\\
0.90281883 0.22984865\\
0.45105876 0.68133515\\
0.80451681 0.66582341\\
0.52075748 0.11651516\\
0.71812397 0.069318238\\
0.56918952 0.8529305\\
0.44348322 0.53651682\\
0.8518447 0.012886306\\
0.75947939 0.88920608\\
0.94975928 0.8660206\\
0.55793851 0.25424693\\
0.59617708 0.15926482\\
0.81620571 0.59436442\\
0.97709235 0.33110008\\
0.70368367 0.86363422\\
0.52206092 0.56762331\\
0.93289706 0.98048127\\
0.71335444 0.7918316\\
0.4496421 0.83302723\\
0.96882014 0.63898657\\
0.3557161 0.66900008\\
0.75533857 0.37981775\\
0.93273619 0.17599564\\
0.94081954 0.790224\\
0.70185317 0.5136085\\
0.45509169 0.15733667\\
0.8511224 0.40775702\\
0.3192963 0.94181515\\
0.199838 0.89664815\\
0.56670978 0.73399632\\
0.52211176 0.39979385\\
0.76991848 0.16930586\\
0.3750558 0.52474549\\
0.8233872 0.64120266\\
0.94915039 0.80346219\\
0.28879765 0.69778483\\
0.88883261 0.46188778\\
0.93309842 0.44535473\\
0.063127896 0.01295783\\
0.99953167 0.87535098\\
0.21198796 0.8352594\\
0.4984098 0.33309507\\
0.2904883 0.88070533\\
0.67275441 0.47968681\\
0.95799112 0.56081673\\
0.76655153 0.61590866\\
0.66612374 0.66189901\\
0.28819335 0.71396084\\
0.81673121 0.5152079\\
0.98548351 0.60586579\\
0.81939293 0.8221174\\
0.62113871 0.31775068\\
0.56022204 0.58769677\\
0.82200759 0.25435426\\
0.26321193 0.8030309\\
0.75363453 0.66784589\\
0.6596448 0.013626353\\
0.60211691 0.45456094\\
0.65950155 0.28215885\\
0.63654691 0.47659189\\
0.17030911 0.98371192\\
0.53960111 0.92234958\\
0.62339418 0.56119625\\
0.67734595 0.7726796\\
0.87683002 0.10617579\\
0.77907925 0.0068577917\\
0.92667831 0.19566205\\
0.67871988 0.78714313\\
0.22715372 0.76170319\\
0.516252 0.90703504\\
0.45820422 0.7585686\\
0.70320334 0.38072994\\
0.58248388 0.33111079\\
0.19323631 0.76719736\\
0.37960413 0.77986778\\
0.77088184 0.80221322\\
0.63819322 0.2027589\\
0.98656686 0.57961489\\
0.5028758 0.66650026\\
0.82802584 0.94251124\\
0.9175572 0.77014853\\
0.81212591 0.86626227\\
0.90826256 0.99094832\\
0.76266647 0.79261032\\
0.72180028 0.44864921\\
0.65163982 0.52435711\\
0.75402267 0.17147322\\
0.66316201 0.13066575\\
0.8834935 0.21878112\\
0.34020345 0.90887473\\
0.8620449 0.65352398\\
0.65661854 0.31343498\\
0.8911828 0.23115882\\
0.48814352 0.41606383\\
0.99264565 0.29879879\\
0.37332594 0.67243639\\
0.53137832 0.93825748\\
0.18131633 0.34314769\\
0.50194434 0.56296257\\
0.6604274 0.16902126\\
0.67365301 0.27889542\\
0.95733018 0.55681466\\
0.56505387 0.23192084\\
0.87021582 0.79272079\\
0.51952869 0.90960045\\
0.19229142 0.92219639\\
0.71568905 0.013266366\\
0.93386482 0.94734264\\
0.52162234 0.92382979\\
0.89520169 0.19899726\\
};
\addlegendentry{class 1};

\addplot [
color=red,
only marks,
mark=+,
mark options={solid}
]
table[row sep=crcr]{
0.50301449 0.33508338\\
0.30620855 0.47115591\\
0.11216371 0.14931034\\
0.44328996 0.13586439\\
0.014668875 0.72578931\\
0.72406171 0.35841852\\
0.2816336 0.28527941\\
0.82886448 0.13471797\\
0.16627012 0.022493293\\
0.39390601 0.26219945\\
0.46080617 0.18033067\\
0.44530705 0.032418564\\
0.087744606 0.73392626\\
0.36629985 0.27602966\\
0.30253382 0.36845815\\
0.014233016 0.5694806\\
0.22190808 0.65861265\\
0.2280389 0.15259362\\
0.1721997 0.19186326\\
0.049046828 0.7720878\\
0.89481276 0.44158549\\
0.28614965 0.48305999\\
0.25120055 0.60810566\\
0.13098247 0.002025561\\
0.84767647 0.21322938\\
0.20927164 0.1034498\\
0.081073725 0.40751498\\
0.56204868 0.052692688\\
0.37489926 0.14997168\\
0.8677957 0.38437405\\
0.37217624 0.31105856\\
0.073690056 0.16853444\\
0.04949328 0.32272443\\
0.12192475 0.41090418\\
0.11706015 0.50552221\\
0.046636146 0.016197484\\
0.59791326 0.83685156\\
0.10158548 0.082612618\\
0.065314507 0.82071701\\
0.23429973 0.19302002\\
0.26421766 0.3087418\\
0.13094482 0.61663323\\
0.095413016 0.68514\\
0.014864337 0.51015303\\
0.017362693 0.96670263\\
0.24403153 0.13020167\\
0.21406286 0.56157937\\
0.60493711 0.90494922\\
0.18336363 0.065034354\\
0.68588997 0.65232469\\
0.0128905 0.0010733656\\
0.31040223 0.54176379\\
0.30729608 0.45133767\\
0.074321493 0.61856262\\
0.070669191 0.015520957\\
0.011930467 0.89085396\\
0.509206 0.50407844\\
0.074289609 0.56456859\\
0.27643472 0.48409788\\
0.31392999 0.47101215\\
0.94770338 0.67676529\\
0.11307983 0.73740432\\
0.15637612 0.5039276\\
0.12211877 0.62908555\\
0.27215744 0.1054805\\
0.41943272 0.14142585\\
0.21299366 0.4569683\\
0.035599967 0.78813305\\
0.081163704 0.28106398\\
0.85056678 0.22478673\\
0.46615498 0.0073288452\\
0.91376324 0.58873933\\
0.22857671 0.54211787\\
0.42219483 0.11888856\\
0.19186575 0.48558916\\
0.11121644 0.95222323\\
0.96916627 0.47865839\\
0.023743871 0.5265224\\
0.026876593 0.19300765\\
0.25067278 0.76754963\\
0.13718931 0.81330552\\
};
\addlegendentry{class -1};

\end{axis}
\end{tikzpicture}%
	\caption{}
	\label{fig:decisionBoundary2100}
\end{figure}



\subsection{Discussion}

This section discusses the results and the convergence of the algorithm that was introduced above. Furthermore it discusses the OR-, AND- and XOR-problems.

\subsubsection{OR-, AND- and XOR-Problem}

We used the same algorithm and evaluation techniques like above to solve this problems. But instead of the \texttt{perceptrondata.dat} we used our own data to demonstrate and visualize this problem. The data X is shown in the Equation~\ref{eq:X}. Each column represents an feature vector in homogeneous coordinates. Each problem is represented with a class label vector. Equation~\ref{eq:prob} shows the vectors for the different problems.

\begin{equation}
X = 
	\begin{bmatrix}
     1  &   1  &   1   &  1\\
     0  &   1  &   0   &  1\\
     0  &   0  &   1   &  1
	\end{bmatrix}
	\label{eq:X}
\end{equation}

\begin{equation}
AND = 
	\begin{bmatrix}
     -1  &   -1  &   -1   &  1
	\end{bmatrix}
	\\
OR = 
	\begin{bmatrix}
     -1  &   1  &   1   &  1
	\end{bmatrix}
	\\
XOR = 
	\begin{bmatrix}
     -1  &   1  &   1   &  -1
	\end{bmatrix}
	\label{eq:prob}
\end{equation}

Figure~\ref{fig:and} shows the results for the AND-Problem and like the OR-Problem Figure~\ref{fig:or} the classification result is $100\%$. The only problem that cannot get solved with the perceptron is the XOR-Problem (see Figure~\ref{fig:xor}) 

There is no linear function that could seperate the 2 classes at the XOR-Problem with a classification rate of $100\%$. So the calculated weight from the algorithm are all zero. (see Equation~\ref{eq:zero})
\begin{equation}
wXOR = 
	\begin{bmatrix}
    0\\0\\0
	\end{bmatrix}
	\label{eq:zero}
\end{equation}
\begin{figure}
	\centering
	\setlength\figureheight{6cm} 
	\setlength\figurewidth{8cm}
	% This file was created by matlab2tikz v0.4.4 running on MATLAB 8.0.
% Copyright (c) 2008--2013, Nico Schlömer <nico.schloemer@gmail.com>
% All rights reserved.
% 
% The latest updates can be retrieved from
%   http://www.mathworks.com/matlabcentral/fileexchange/22022-matlab2tikz
% where you can also make suggestions and rate matlab2tikz.
% 
%
% defining custom colors
\definecolor{mycolor1}{rgb}{1,0,1}%
%
\begin{tikzpicture}

\begin{axis}[%
width=\figurewidth,
height=\figureheight,
colormap/jet,
unbounded coords=jump,
scale only axis,
xmin=-0.1,
xmax=1.1,
xlabel={${\text{x}}$},
ymin=-0.1,
ymax=1.1,
ylabel={${\text{y}}$},
title={AND-Problem},
legend style={at={(0.5,-0.03)},anchor=north,draw=black,fill=white,legend cell align=left}
]

\addplot[area legend,solid,draw=mycolor1]
table[row sep=crcr]{
x y\\
-6.28318530717959 5.52212353811972 \\
-6.23271795531469 5.48847863687646 \\
-6.21356151601187 5.47570767734125 \\
-6.18225060344979 5.4548337356332 \\
-6.13786048821452 5.42524032547635 \\
-6.1317832515849 5.42118883438993 \\
-6.08131589972 5.38754393314667 \\
-6.06215946041718 5.37477297361145 \\
-6.03084854785511 5.3538990319034 \\
-5.98645843261984 5.32430562174656 \\
-5.98038119599021 5.32025413066014 \\
-5.92991384412531 5.28660922941688 \\
-5.91075740482249 5.27383826988166 \\
-5.87944649226042 5.25296432817361 \\
-5.83505637702515 5.22337091801676 \\
-5.82897914039552 5.21931942693035 \\
-5.77851178853062 5.18567452568708 \\
-5.7593553492278 5.17290356615187 \\
-5.72804443666573 5.15202962444382 \\
-5.68365432143046 5.12243621428697 \\
-5.67757708480083 5.11838472320055 \\
-5.62710973293594 5.08473982195729 \\
-5.60795329363311 5.07196886242208 \\
-5.57664238107104 5.05109492071403 \\
-5.53225226583577 5.02150151055718 \\
-5.52617502920614 5.01745001947076 \\
-5.47570767734125 4.9838051182275 \\
-5.45655123803843 4.97103415869228 \\
-5.42524032547635 4.95016021698423 \\
-5.38085021024108 4.92056680682739 \\
-5.37477297361145 4.91651531574097 \\
-5.32430562174656 4.88287041449771 \\
-5.30514918244374 4.87009945496249 \\
-5.27383826988166 4.84922551325444 \\
-5.22944815464639 4.81963210309759 \\
-5.22337091801676 4.81558061201118 \\
-5.17290356615187 4.78193571076791 \\
-5.15374712684905 4.7691647512327 \\
-5.12243621428697 4.74829080952465 \\
-5.0780460990517 4.7186973993678 \\
-5.07196886242208 4.71464590828138 \\
-5.02150151055718 4.68100100703812 \\
-5.00234507125436 4.6682300475029 \\
-4.97103415869228 4.64735610579485 \\
-4.92664404345701 4.61776269563801 \\
-4.92056680682739 4.61371120455159 \\
-4.87009945496249 4.58006630330833 \\
-4.85094301565967 4.56729534377311 \\
-4.81963210309759 4.54642140206506 \\
-4.77524198786233 4.51682799190822 \\
-4.7691647512327 4.5127765008218 \\
-4.7186973993678 4.47913159957853 \\
-4.69954096006498 4.46636064004332 \\
-4.66823004750291 4.44548669833527 \\
-4.62383993226764 4.41589328817842 \\
-4.61776269563801 4.41184179709201 \\
-4.56729534377311 4.37819689584874 \\
-4.54813890447029 4.36542593631353 \\
-4.51682799190822 4.34455199460548 \\
-4.47243787667295 4.31495858444863 \\
-4.46636064004332 4.31090709336221 \\
-4.41589328817842 4.27726219211895 \\
-4.39673684887561 4.26449123258374 \\
-4.36542593631353 4.24361729087568 \\
-4.32103582107826 4.21402388071884 \\
-4.31495858444863 4.20997238963242 \\
-4.26449123258374 4.17632748838916 \\
-4.24533479328091 4.16355652885394 \\
-4.21402388071884 4.14268258714589 \\
-4.16963376548357 4.11308917698905 \\
-4.16355652885394 4.10903768590263 \\
-4.11308917698905 4.07539278465936 \\
-4.09393273768623 4.06262182512415 \\
-4.06262182512415 4.0417478834161 \\
-4.01823170988888 4.01215447325925 \\
-4.01215447325925 4.00810298217284 \\
-3.96168712139436 3.97445808092957 \\
-3.94253068209154 3.96168712139436 \\
-3.91121976952946 3.94081317968631 \\
-3.86682965429419 3.91121976952946 \\
-3.86075241766456 3.90716827844304 \\
-3.81028506579967 3.87352337719978 \\
-3.79112862649685 3.86075241766457 \\
-3.75981771393477 3.83987847595652 \\
-3.71542759869951 3.81028506579967 \\
-3.70935036206988 3.80623357471325 \\
-3.65888301020498 3.77258867346999 \\
-3.63972657090216 3.75981771393477 \\
-3.60841565834008 3.73894377222672 \\
-3.56402554310481 3.70935036206988 \\
-3.55794830647519 3.70529887098346 \\
-3.50748095461029 3.67165396974019 \\
-3.48832451530747 3.65888301020498 \\
-3.45701360274539 3.63800906849693 \\
-3.41262348751013 3.60841565834008 \\
-3.4065462508805 3.60436416725367 \\
-3.3560788990156 3.5707192660104 \\
-3.33692245971278 3.55794830647519 \\
-3.30561154715071 3.53707436476714 \\
-3.26122143191544 3.50748095461029 \\
-3.25514419528581 3.50342946352387 \\
-3.20467684342091 3.46978456228061 \\
-3.18552040411809 3.45701360274539 \\
-3.15420949155602 3.43613966103734 \\
-3.10981937632075 3.4065462508805 \\
-3.10374213969112 3.40249475979408 \\
-3.05327478782622 3.36884985855082 \\
-3.0341183485234 3.3560788990156 \\
-3.00280743596133 3.33520495730755 \\
-2.95841732072606 3.30561154715071 \\
-2.95234008409643 3.30156005606429 \\
-2.90187273223154 3.26791515482102 \\
-2.88271629292872 3.25514419528581 \\
-2.85140538036664 3.23427025357776 \\
-2.80701526513137 3.20467684342091 \\
-2.80093802850174 3.2006253523345 \\
-2.75047067663685 3.16698045109123 \\
-2.73131423733403 3.15420949155602 \\
-2.70000332477195 3.13333554984797 \\
-2.65561320953668 3.10374213969112 \\
-2.64953597290705 3.0996906486047 \\
-2.59906862104216 3.06604574736144 \\
-2.57991218173934 3.05327478782622 \\
-2.54860126917726 3.03240084611817 \\
-2.50421115394199 3.00280743596133 \\
-2.49813391731237 2.99875594487491 \\
-2.44766656544747 2.96511104363165 \\
-2.42851012614465 2.95234008409643 \\
-2.39719921358257 2.93146614238838 \\
-2.3528090983473 2.90187273223154 \\
-2.34673186171768 2.89782124114512 \\
-2.29626450985278 2.86417633990185 \\
-2.27710807054996 2.85140538036664 \\
-2.24579715798788 2.83053143865859 \\
-2.20140704275262 2.80093802850174 \\
-2.19532980612299 2.79688653741533 \\
-2.14486245425809 2.76324163617206 \\
-2.12570601495527 2.75047067663685 \\
-2.09439510239319 2.7295967349288 \\
-2.05000498715793 2.70000332477195 \\
-2.0439277505283 2.69595183368553 \\
-1.9934603986634 2.66230693244227 \\
-1.97430395936058 2.64953597290705 \\
-1.94299304679851 2.628662031199 \\
-1.89860293156324 2.59906862104216 \\
-1.89252569493361 2.59501712995574 \\
-1.84205834306871 2.56137222871248 \\
-1.82290190376589 2.54860126917726 \\
-1.79159099120382 2.52772732746921 \\
-1.74720087596855 2.49813391731237 \\
-1.74112363933892 2.49408242622595 \\
-1.69065628747403 2.46043752498268 \\
-1.6714998481712 2.44766656544747 \\
-1.64018893560913 2.42679262373942 \\
-1.59579882037386 2.39719921358257 \\
-1.58972158374423 2.39314772249616 \\
-1.53925423187934 2.35950282125289 \\
-1.52009779257652 2.34673186171768 \\
-1.48878688001444 2.32585792000963 \\
-1.44439676477917 2.29626450985278 \\
-1.43831952814954 2.29221301876636 \\
-1.38785217628465 2.2585681175231 \\
-1.36869573698183 2.24579715798788 \\
-1.33738482441975 2.22492321627983 \\
-1.29299470918448 2.19532980612299 \\
-1.28691747255485 2.19127831503657 \\
-1.23645012068996 2.15763341379331 \\
-1.21729368138714 2.14486245425809 \\
-1.18598276882506 2.12398851255004 \\
-1.14159265358979 2.0943951023932 \\
-1.13551541696017 2.09034361130678 \\
-1.08504806509527 2.05669871006351 \\
-1.06589162579245 2.0439277505283 \\
-1.03458071323037 2.02305380882025 \\
-0.990190597995102 1.9934603986634 \\
-0.984113361365477 1.98940890757698 \\
-0.933646009500581 1.95576400633372 \\
-0.914489570197759 1.94299304679851 \\
-0.883178657635685 1.92211910509046 \\
-0.838788542400416 1.89252569493361 \\
-0.832711305770789 1.88847420384719 \\
-0.782243953905892 1.85482930260393 \\
-0.763087514603071 1.84205834306871 \\
-0.731776602040996 1.82118440136066 \\
-0.687386486805726 1.79159099120382 \\
-0.681309250176099 1.7875395001174 \\
-0.630841898311203 1.75389459887414 \\
-0.611685459008382 1.74112363933892 \\
-0.580374546446308 1.72024969763087 \\
-0.535984431211039 1.69065628747403 \\
-0.529907194581411 1.68660479638761 \\
-0.479439842716515 1.65295989514434 \\
-0.460283403413692 1.64018893560913 \\
-0.428972490851618 1.61931499390108 \\
-0.384582375616349 1.58972158374423 \\
-0.378505138986722 1.58567009265781 \\
-0.328037787121826 1.55202519141455 \\
-0.308881347819005 1.53925423187934 \\
-0.27757043525693 1.51838029017129 \\
-0.23318032002166 1.48878688001444 \\
-0.227103083392033 1.48473538892802 \\
-0.176635731527137 1.45109048768476 \\
-0.157479292224317 1.43831952814954 \\
-0.12616837966224 1.41744558644149 \\
-0.0817782644269703 1.38785217628465 \\
-0.0757010277973444 1.38380068519823 \\
-0.0252336759324487 1.35015578395497 \\
-0.00607723662962716 1.33738482441975 \\
0.0252336759324479 1.3165108827117 \\
0.0696237911677177 1.28691747255485 \\
0.0757010277973444 1.28286598146844 \\
0.126168379662241 1.24922108022517 \\
0.145324818965061 1.23645012068996 \\
0.176635731527137 1.21557617898191 \\
0.221025846762407 1.18598276882506 \\
0.227103083392033 1.18193127773864 \\
0.277570435256929 1.14828637649538 \\
0.29672687455975 1.13551541696017 \\
0.328037787121827 1.11464147525212 \\
0.372427902357096 1.08504806509527 \\
0.378505138986722 1.08099657400885 \\
0.428972490851619 1.04735167276559 \\
0.44812893015444 1.03458071323037 \\
0.479439842716515 1.01370677152232 \\
0.523829957951784 0.984113361365478 \\
0.52990719458141 0.980061870279059 \\
0.580374546446308 0.946416969035796 \\
0.599530985749128 0.933646009500581 \\
0.630841898311203 0.912772067792531 \\
0.675232013546472 0.883178657635685 \\
0.6813092501761 0.879127166549267 \\
0.731776602040996 0.845482265306003 \\
0.750933041343817 0.832711305770788 \\
0.782243953905892 0.811837364062739 \\
0.826634069141161 0.782243953905892 \\
0.832711305770788 0.778192462819474 \\
0.883178657635685 0.74454756157621 \\
0.902335096938507 0.731776602040996 \\
0.933646009500581 0.710902660332946 \\
0.97803612473585 0.6813092501761 \\
0.984113361365478 0.677257759089682 \\
1.03458071323037 0.643612857846418 \\
1.05373715253319 0.630841898311203 \\
1.08504806509527 0.609967956603154 \\
1.12943818033054 0.580374546446308 \\
1.13551541696017 0.576323055359889 \\
1.18598276882506 0.542678154116625 \\
1.20513920812788 0.52990719458141 \\
1.23645012068996 0.509033252873361 \\
1.28084023592523 0.479439842716515 \\
1.28691747255485 0.475388351630096 \\
1.33738482441975 0.441743450386833 \\
1.35654126372257 0.428972490851619 \\
1.38785217628465 0.408098549143569 \\
1.43224229151992 0.378505138986722 \\
1.43831952814954 0.374453647900304 \\
1.48878688001444 0.340808746657039 \\
1.50794331931726 0.328037787121827 \\
1.53925423187934 0.307163845413775 \\
1.58364434711461 0.277570435256929 \\
1.58972158374423 0.273518944170512 \\
1.64018893560913 0.239874042927248 \\
1.65934537491195 0.227103083392033 \\
1.69065628747403 0.206229141683982 \\
1.73504640270929 0.176635731527137 \\
1.74112363933892 0.172584240440719 \\
1.79159099120382 0.138939339197455 \\
1.81074743050664 0.126168379662241 \\
1.84205834306871 0.10529443795419 \\
1.88644845830398 0.0757010277973444 \\
1.89252569493361 0.0716495367109269 \\
1.94299304679851 0.038004635467663 \\
1.96214948610133 0.0252336759324479 \\
1.9934603986634 0.00435973422439904 \\
2.03785051389867 -0.0252336759324487 \\
2.0439277505283 -0.0292851670188662 \\
2.0943951023932 -0.062930068262131 \\
2.11355154169602 -0.0757010277973444 \\
2.14486245425809 -0.096574969505395 \\
2.18925256949336 -0.12616837966224 \\
2.19532980612299 -0.130219870748658 \\
2.24579715798788 -0.163864771991922 \\
2.26495359729071 -0.176635731527137 \\
2.29626450985278 -0.197509673235188 \\
2.34065462508805 -0.227103083392033 \\
2.34673186171768 -0.231154574478451 \\
2.39719921358257 -0.264799475721715 \\
2.41635565288539 -0.27757043525693 \\
2.44766656544747 -0.298444376964979 \\
2.49205668068274 -0.328037787121826 \\
2.49813391731237 -0.332089278208244 \\
2.54860126917726 -0.365734179451508 \\
2.56775770848008 -0.378505138986722 \\
2.59906862104216 -0.399379080694771 \\
2.64345873627743 -0.428972490851618 \\
2.64953597290705 -0.433023981938036 \\
2.70000332477195 -0.466668883181302 \\
2.71915976407477 -0.479439842716515 \\
2.75047067663685 -0.500313784424566 \\
2.79486079187212 -0.529907194581411 \\
2.80093802850174 -0.533958685667829 \\
2.85140538036664 -0.567603586911093 \\
2.87056181966946 -0.580374546446308 \\
2.90187273223154 -0.601248488154357 \\
2.94626284746681 -0.630841898311203 \\
2.95234008409643 -0.634893389397622 \\
3.00280743596133 -0.668538290640886 \\
3.02196387526415 -0.681309250176099 \\
3.05327478782622 -0.702183191884149 \\
3.09766490306149 -0.731776602040996 \\
3.10374213969112 -0.735828093127413 \\
3.15420949155602 -0.769472994370679 \\
3.17336593085884 -0.782243953905892 \\
3.20467684342091 -0.803117895613943 \\
3.24906695865618 -0.832711305770789 \\
3.25514419528581 -0.836762796857207 \\
3.30561154715071 -0.87040769810047 \\
3.32476798645353 -0.883178657635685 \\
3.3560788990156 -0.904052599343734 \\
3.40046901425087 -0.933646009500581 \\
3.4065462508805 -0.937697500586999 \\
3.45701360274539 -0.971342401830263 \\
3.47617004204821 -0.984113361365477 \\
3.50748095461029 -1.00498730307353 \\
3.55187106984556 -1.03458071323037 \\
3.55794830647519 -1.03863220431679 \\
3.60841565834008 -1.07227710556006 \\
3.62757209764291 -1.08504806509527 \\
3.65888301020498 -1.10592200680332 \\
3.70327312544025 -1.13551541696017 \\
3.70935036206988 -1.13956690804658 \\
3.75981771393477 -1.17321180928985 \\
3.77897415323759 -1.18598276882506 \\
3.81028506579967 -1.20685671053311 \\
3.85467518103494 -1.23645012068996 \\
3.86075241766457 -1.24050161177638 \\
3.91121976952946 -1.27414651301964 \\
3.93037620883228 -1.28691747255485 \\
3.96168712139436 -1.3077914142629 \\
4.00607723662963 -1.33738482441975 \\
4.01215447325925 -1.34143631550617 \\
4.06262182512415 -1.37508121674943 \\
4.08177826442697 -1.38785217628465 \\
4.11308917698905 -1.4087261179927 \\
4.15747929222432 -1.43831952814954 \\
4.16355652885394 -1.44237101923596 \\
4.21402388071884 -1.47601592047923 \\
4.23318032002166 -1.48878688001444 \\
4.26449123258374 -1.50966082172249 \\
4.308881347819 -1.53925423187934 \\
4.31495858444863 -1.54330572296576 \\
4.36542593631353 -1.57695062420902 \\
4.38458237561635 -1.58972158374423 \\
4.41589328817842 -1.61059552545228 \\
4.46028340341369 -1.64018893560913 \\
4.46636064004332 -1.64424042669555 \\
4.51682799190822 -1.67788532793881 \\
4.53598443121104 -1.69065628747403 \\
4.56729534377311 -1.71153022918208 \\
4.61168545900838 -1.74112363933892 \\
4.61776269563801 -1.74517513042534 \\
4.6682300475029 -1.7788200316686 \\
4.68738648680573 -1.79159099120382 \\
4.7186973993678 -1.81246493291187 \\
4.76308751460307 -1.84205834306871 \\
4.7691647512327 -1.84610983415513 \\
4.81963210309759 -1.8797547353984 \\
4.83878854240041 -1.89252569493361 \\
4.87009945496249 -1.91339963664166 \\
4.91448957019776 -1.94299304679851 \\
4.92056680682739 -1.94704453788493 \\
4.97103415869228 -1.98068943912819 \\
4.9901905979951 -1.9934603986634 \\
5.02150151055718 -2.01433434037145 \\
5.06589162579245 -2.0439277505283 \\
5.07196886242208 -2.04797924161472 \\
5.12243621428697 -2.08162414285798 \\
5.14159265358979 -2.09439510239319 \\
5.17290356615187 -2.11526904410125 \\
5.21729368138714 -2.14486245425809 \\
5.22337091801676 -2.14891394534451 \\
5.27383826988166 -2.18255884658777 \\
5.29299470918448 -2.19532980612299 \\
5.32430562174656 -2.21620374783104 \\
5.36869573698183 -2.24579715798788 \\
5.37477297361145 -2.2498486490743 \\
5.42524032547635 -2.28349355031757 \\
5.44439676477917 -2.29626450985278 \\
5.47570767734125 -2.31713845156083 \\
5.52009779257652 -2.34673186171768 \\
5.52617502920614 -2.35078335280409 \\
5.57664238107104 -2.38442825404736 \\
5.59579882037386 -2.39719921358257 \\
5.62710973293594 -2.41807315529062 \\
5.6714998481712 -2.44766656544747 \\
5.67757708480083 -2.45171805653389 \\
5.72804443666573 -2.48536295777715 \\
5.74720087596855 -2.49813391731237 \\
5.77851178853062 -2.51900785902041 \\
5.82290190376589 -2.54860126917726 \\
5.82897914039552 -2.55265276026368 \\
5.87944649226042 -2.58629766150694 \\
5.89860293156324 -2.59906862104216 \\
5.92991384412531 -2.61994256275021 \\
5.97430395936058 -2.64953597290705 \\
5.98038119599021 -2.65358746399347 \\
6.03084854785511 -2.68723236523674 \\
6.05000498715793 -2.70000332477195 \\
6.08131589972 -2.72087726648 \\
6.12570601495527 -2.75047067663685 \\
6.1317832515849 -2.75452216772327 \\
6.18225060344979 -2.78816706896653 \\
6.20140704275261 -2.80093802850174 \\
6.23271795531469 -2.82181197020979 \\
6.27710807054996 -2.85140538036664 \\
6.28318530717959 -2.85545687145306 \\
NaN NaN \\
};

\addlegendentry{decision boundary};

\addplot [
color=green,
only marks,
mark=asterisk,
mark options={solid}
]
table[row sep=crcr]{
1 1\\
};
\addlegendentry{class 1};

\addplot [
color=red,
only marks,
mark=+,
mark options={solid}
]
table[row sep=crcr]{
0 0\\
1 0\\
0 1\\
};
\addlegendentry{class -1};

\end{axis}
\end{tikzpicture}%
	\caption{}
	\label{fig:and}
\end{figure}
\begin{figure}
	\centering
	\setlength\figureheight{6cm} 
	\setlength\figurewidth{8cm}
	% This file was created by matlab2tikz v0.4.4 running on MATLAB 8.0.
% Copyright (c) 2008--2013, Nico Schlömer <nico.schloemer@gmail.com>
% All rights reserved.
% 
% The latest updates can be retrieved from
%   http://www.mathworks.com/matlabcentral/fileexchange/22022-matlab2tikz
% where you can also make suggestions and rate matlab2tikz.
% 
%
% defining custom colors
\definecolor{mycolor1}{rgb}{1,0,1}%
%
\begin{tikzpicture}

\begin{axis}[%
width=\figurewidth,
height=\figureheight,
colormap/jet,
unbounded coords=jump,
scale only axis,
xmin=-0.1,
xmax=1.1,
xlabel={${\text{x}}$},
ymin=-0.1,
ymax=1.1,
ylabel={${\text{y}}$},
title={OR-Problem},
legend style={at={(0.5,-0.03)},anchor=north,draw=black,fill=white,legend cell align=left}
]

\addplot[area legend,solid,draw=mycolor1]
table[row sep=crcr]{
x y\\
-5.78318530717959 6.28318530717959 \\
-5.77851178853062 6.27851178853062 \\
-5.73271795531469 6.23271795531469 \\
-5.72804443666573 6.22804443666573 \\
-5.68225060344979 6.18225060344979 \\
-5.67757708480083 6.17757708480083 \\
-5.6317832515849 6.1317832515849 \\
-5.62710973293594 6.12710973293594 \\
-5.58131589972 6.08131589972 \\
-5.57664238107104 6.07664238107104 \\
-5.53084854785511 6.03084854785511 \\
-5.52617502920614 6.02617502920614 \\
-5.48038119599021 5.98038119599021 \\
-5.47570767734125 5.97570767734125 \\
-5.42991384412531 5.92991384412531 \\
-5.42524032547635 5.92524032547635 \\
-5.37944649226042 5.87944649226042 \\
-5.37477297361145 5.87477297361145 \\
-5.32897914039552 5.82897914039552 \\
-5.32430562174656 5.82430562174656 \\
-5.27851178853062 5.77851178853062 \\
-5.27383826988166 5.77383826988166 \\
-5.22804443666573 5.72804443666573 \\
-5.22337091801676 5.72337091801676 \\
-5.17757708480083 5.67757708480083 \\
-5.17290356615187 5.67290356615187 \\
-5.12710973293594 5.62710973293594 \\
-5.12243621428697 5.62243621428697 \\
-5.07664238107104 5.57664238107104 \\
-5.07196886242208 5.57196886242208 \\
-5.02617502920614 5.52617502920614 \\
-5.02150151055718 5.52150151055718 \\
-4.97570767734125 5.47570767734125 \\
-4.97103415869228 5.47103415869228 \\
-4.92524032547635 5.42524032547635 \\
-4.92056680682739 5.42056680682739 \\
-4.87477297361145 5.37477297361145 \\
-4.87009945496249 5.37009945496249 \\
-4.82430562174656 5.32430562174656 \\
-4.81963210309759 5.31963210309759 \\
-4.77383826988166 5.27383826988166 \\
-4.7691647512327 5.2691647512327 \\
-4.72337091801676 5.22337091801676 \\
-4.7186973993678 5.2186973993678 \\
-4.67290356615187 5.17290356615187 \\
-4.66823004750291 5.16823004750291 \\
-4.62243621428697 5.12243621428697 \\
-4.61776269563801 5.11776269563801 \\
-4.57196886242208 5.07196886242208 \\
-4.56729534377311 5.06729534377311 \\
-4.52150151055718 5.02150151055718 \\
-4.51682799190822 5.01682799190822 \\
-4.47103415869228 4.97103415869228 \\
-4.46636064004332 4.96636064004332 \\
-4.42056680682739 4.92056680682739 \\
-4.41589328817842 4.91589328817842 \\
-4.37009945496249 4.87009945496249 \\
-4.36542593631353 4.86542593631353 \\
-4.31963210309759 4.81963210309759 \\
-4.31495858444863 4.81495858444863 \\
-4.2691647512327 4.7691647512327 \\
-4.26449123258374 4.76449123258374 \\
-4.2186973993678 4.7186973993678 \\
-4.21402388071884 4.71402388071884 \\
-4.1682300475029 4.6682300475029 \\
-4.16355652885394 4.66355652885394 \\
-4.11776269563801 4.61776269563801 \\
-4.11308917698905 4.61308917698905 \\
-4.06729534377311 4.56729534377311 \\
-4.06262182512415 4.56262182512415 \\
-4.01682799190822 4.51682799190822 \\
-4.01215447325925 4.51215447325925 \\
-3.96636064004332 4.46636064004332 \\
-3.96168712139436 4.46168712139436 \\
-3.91589328817842 4.41589328817842 \\
-3.91121976952946 4.41121976952946 \\
-3.86542593631353 4.36542593631353 \\
-3.86075241766456 4.36075241766456 \\
-3.81495858444863 4.31495858444863 \\
-3.81028506579967 4.31028506579967 \\
-3.76449123258374 4.26449123258374 \\
-3.75981771393477 4.25981771393477 \\
-3.71402388071884 4.21402388071884 \\
-3.70935036206988 4.20935036206988 \\
-3.66355652885394 4.16355652885394 \\
-3.65888301020498 4.15888301020498 \\
-3.61308917698905 4.11308917698905 \\
-3.60841565834008 4.10841565834008 \\
-3.56262182512415 4.06262182512415 \\
-3.55794830647519 4.05794830647519 \\
-3.51215447325925 4.01215447325925 \\
-3.50748095461029 4.00748095461029 \\
-3.46168712139436 3.96168712139436 \\
-3.45701360274539 3.9570136027454 \\
-3.41121976952946 3.91121976952946 \\
-3.4065462508805 3.9065462508805 \\
-3.36075241766457 3.86075241766457 \\
-3.3560788990156 3.8560788990156 \\
-3.31028506579967 3.81028506579967 \\
-3.30561154715071 3.80561154715071 \\
-3.25981771393477 3.75981771393477 \\
-3.25514419528581 3.75514419528581 \\
-3.20935036206988 3.70935036206988 \\
-3.20467684342091 3.70467684342091 \\
-3.15888301020498 3.65888301020498 \\
-3.15420949155602 3.65420949155602 \\
-3.10841565834008 3.60841565834008 \\
-3.10374213969112 3.60374213969112 \\
-3.05794830647519 3.55794830647519 \\
-3.05327478782622 3.55327478782622 \\
-3.00748095461029 3.50748095461029 \\
-3.00280743596133 3.50280743596133 \\
-2.95701360274539 3.45701360274539 \\
-2.95234008409643 3.45234008409643 \\
-2.9065462508805 3.4065462508805 \\
-2.90187273223154 3.40187273223154 \\
-2.8560788990156 3.3560788990156 \\
-2.85140538036664 3.35140538036664 \\
-2.80561154715071 3.30561154715071 \\
-2.80093802850174 3.30093802850174 \\
-2.75514419528581 3.25514419528581 \\
-2.75047067663685 3.25047067663685 \\
-2.70467684342091 3.20467684342091 \\
-2.70000332477195 3.20000332477195 \\
-2.65420949155602 3.15420949155602 \\
-2.64953597290705 3.14953597290705 \\
-2.60374213969112 3.10374213969112 \\
-2.59906862104216 3.09906862104216 \\
-2.55327478782622 3.05327478782622 \\
-2.54860126917726 3.04860126917726 \\
-2.50280743596133 3.00280743596133 \\
-2.49813391731237 2.99813391731237 \\
-2.45234008409643 2.95234008409643 \\
-2.44766656544747 2.94766656544747 \\
-2.40187273223154 2.90187273223154 \\
-2.39719921358257 2.89719921358257 \\
-2.35140538036664 2.85140538036664 \\
-2.34673186171768 2.84673186171768 \\
-2.30093802850174 2.80093802850174 \\
-2.29626450985278 2.79626450985278 \\
-2.25047067663685 2.75047067663685 \\
-2.24579715798788 2.74579715798788 \\
-2.20000332477195 2.70000332477195 \\
-2.19532980612299 2.69532980612299 \\
-2.14953597290706 2.64953597290705 \\
-2.14486245425809 2.64486245425809 \\
-2.09906862104216 2.59906862104216 \\
-2.09439510239319 2.5943951023932 \\
-2.04860126917726 2.54860126917726 \\
-2.0439277505283 2.5439277505283 \\
-1.99813391731237 2.49813391731237 \\
-1.9934603986634 2.4934603986634 \\
-1.94766656544747 2.44766656544747 \\
-1.94299304679851 2.44299304679851 \\
-1.89719921358257 2.39719921358257 \\
-1.89252569493361 2.39252569493361 \\
-1.84673186171768 2.34673186171768 \\
-1.84205834306871 2.34205834306872 \\
-1.79626450985278 2.29626450985278 \\
-1.79159099120382 2.29159099120382 \\
-1.74579715798788 2.24579715798788 \\
-1.74112363933892 2.24112363933892 \\
-1.69532980612299 2.19532980612299 \\
-1.69065628747403 2.19065628747403 \\
-1.64486245425809 2.14486245425809 \\
-1.64018893560913 2.14018893560913 \\
-1.5943951023932 2.0943951023932 \\
-1.58972158374423 2.08972158374423 \\
-1.5439277505283 2.0439277505283 \\
-1.53925423187934 2.03925423187934 \\
-1.4934603986634 1.9934603986634 \\
-1.48878688001444 1.98878688001444 \\
-1.44299304679851 1.94299304679851 \\
-1.43831952814954 1.93831952814954 \\
-1.39252569493361 1.89252569493361 \\
-1.38785217628465 1.88785217628465 \\
-1.34205834306871 1.84205834306871 \\
-1.33738482441975 1.83738482441975 \\
-1.29159099120382 1.79159099120382 \\
-1.28691747255485 1.78691747255486 \\
-1.24112363933892 1.74112363933892 \\
-1.23645012068996 1.73645012068996 \\
-1.19065628747403 1.69065628747403 \\
-1.18598276882506 1.68598276882506 \\
-1.14018893560913 1.64018893560913 \\
-1.13551541696017 1.63551541696017 \\
-1.08972158374423 1.58972158374423 \\
-1.08504806509527 1.58504806509527 \\
-1.03925423187934 1.53925423187934 \\
-1.03458071323037 1.53458071323037 \\
-0.98878688001444 1.48878688001444 \\
-0.984113361365477 1.48411336136548 \\
-0.938319528149545 1.43831952814954 \\
-0.933646009500581 1.43364600950058 \\
-0.887852176284647 1.38785217628465 \\
-0.883178657635685 1.38317865763569 \\
-0.837384824419751 1.33738482441975 \\
-0.832711305770789 1.33271130577079 \\
-0.786917472554855 1.28691747255485 \\
-0.782243953905892 1.28224395390589 \\
-0.736450120689959 1.23645012068996 \\
-0.731776602040996 1.231776602041 \\
-0.685982768825063 1.18598276882506 \\
-0.681309250176099 1.1813092501761 \\
-0.635515416960167 1.13551541696017 \\
-0.630841898311203 1.1308418983112 \\
-0.585048065095269 1.08504806509527 \\
-0.580374546446308 1.08037454644631 \\
-0.534580713230374 1.03458071323037 \\
-0.529907194581411 1.02990719458141 \\
-0.484113361365478 0.984113361365478 \\
-0.479439842716515 0.979439842716514 \\
-0.433646009500581 0.933646009500581 \\
-0.428972490851618 0.928972490851618 \\
-0.383178657635686 0.883178657635685 \\
-0.378505138986722 0.878505138986722 \\
-0.332711305770788 0.832711305770788 \\
-0.328037787121826 0.828037787121826 \\
-0.282243953905893 0.782243953905892 \\
-0.27757043525693 0.77757043525693 \\
-0.231776602040996 0.731776602040996 \\
-0.227103083392033 0.727103083392034 \\
-0.1813092501761 0.6813092501761 \\
-0.176635731527137 0.676635731527137 \\
-0.130841898311204 0.630841898311203 \\
-0.12616837966224 0.62616837966224 \\
-0.0803745464463078 0.580374546446308 \\
-0.0757010277973444 0.575701027797344 \\
-0.0299071945814099 0.52990719458141 \\
-0.0252336759324487 0.525233675932448 \\
0.0205601572834851 0.479439842716515 \\
0.0252336759324479 0.474766324067553 \\
0.071027509148381 0.428972490851619 \\
0.0757010277973444 0.424298972202655 \\
0.121494861013278 0.378505138986722 \\
0.126168379662241 0.373831620337759 \\
0.171962212878173 0.328037787121827 \\
0.176635731527137 0.323364268472863 \\
0.222429564743071 0.277570435256929 \\
0.227103083392033 0.272896916607967 \\
0.272896916607967 0.227103083392033 \\
0.277570435256929 0.222429564743071 \\
0.323364268472863 0.176635731527137 \\
0.328037787121827 0.171962212878173 \\
0.373831620337759 0.126168379662241 \\
0.378505138986722 0.121494861013278 \\
0.424298972202655 0.0757010277973444 \\
0.428972490851619 0.071027509148381 \\
0.474766324067553 0.0252336759324479 \\
0.479439842716515 0.0205601572834851 \\
0.525233675932448 -0.0252336759324487 \\
0.52990719458141 -0.0299071945814099 \\
0.575701027797344 -0.0757010277973444 \\
0.580374546446308 -0.0803745464463078 \\
0.62616837966224 -0.12616837966224 \\
0.630841898311203 -0.130841898311204 \\
0.676635731527137 -0.176635731527137 \\
0.6813092501761 -0.1813092501761 \\
0.727103083392034 -0.227103083392033 \\
0.731776602040996 -0.231776602040996 \\
0.77757043525693 -0.27757043525693 \\
0.782243953905892 -0.282243953905893 \\
0.828037787121826 -0.328037787121826 \\
0.832711305770788 -0.332711305770788 \\
0.878505138986722 -0.378505138986722 \\
0.883178657635685 -0.383178657635686 \\
0.928972490851618 -0.428972490851618 \\
0.933646009500581 -0.433646009500581 \\
0.979439842716514 -0.479439842716515 \\
0.984113361365478 -0.484113361365478 \\
1.02990719458141 -0.529907194581411 \\
1.03458071323037 -0.534580713230374 \\
1.08037454644631 -0.580374546446308 \\
1.08504806509527 -0.585048065095269 \\
1.1308418983112 -0.630841898311203 \\
1.13551541696017 -0.635515416960167 \\
1.1813092501761 -0.681309250176099 \\
1.18598276882506 -0.685982768825063 \\
1.231776602041 -0.731776602040996 \\
1.23645012068996 -0.736450120689959 \\
1.28224395390589 -0.782243953905892 \\
1.28691747255485 -0.786917472554855 \\
1.33271130577079 -0.832711305770789 \\
1.33738482441975 -0.837384824419751 \\
1.38317865763569 -0.883178657635685 \\
1.38785217628465 -0.887852176284647 \\
1.43364600950058 -0.933646009500581 \\
1.43831952814954 -0.938319528149545 \\
1.48411336136548 -0.984113361365477 \\
1.48878688001444 -0.98878688001444 \\
1.53458071323037 -1.03458071323037 \\
1.53925423187934 -1.03925423187934 \\
1.58504806509527 -1.08504806509527 \\
1.58972158374423 -1.08972158374423 \\
1.63551541696017 -1.13551541696017 \\
1.64018893560913 -1.14018893560913 \\
1.68598276882506 -1.18598276882506 \\
1.69065628747403 -1.19065628747403 \\
1.73645012068996 -1.23645012068996 \\
1.74112363933892 -1.24112363933892 \\
1.78691747255486 -1.28691747255485 \\
1.79159099120382 -1.29159099120382 \\
1.83738482441975 -1.33738482441975 \\
1.84205834306871 -1.34205834306871 \\
1.88785217628465 -1.38785217628465 \\
1.89252569493361 -1.39252569493361 \\
1.93831952814954 -1.43831952814954 \\
1.94299304679851 -1.44299304679851 \\
1.98878688001444 -1.48878688001444 \\
1.9934603986634 -1.4934603986634 \\
2.03925423187934 -1.53925423187934 \\
2.0439277505283 -1.5439277505283 \\
2.08972158374423 -1.58972158374423 \\
2.0943951023932 -1.5943951023932 \\
2.14018893560913 -1.64018893560913 \\
2.14486245425809 -1.64486245425809 \\
2.19065628747403 -1.69065628747403 \\
2.19532980612299 -1.69532980612299 \\
2.24112363933892 -1.74112363933892 \\
2.24579715798788 -1.74579715798788 \\
2.29159099120382 -1.79159099120382 \\
2.29626450985278 -1.79626450985278 \\
2.34205834306872 -1.84205834306871 \\
2.34673186171768 -1.84673186171768 \\
2.39252569493361 -1.89252569493361 \\
2.39719921358257 -1.89719921358257 \\
2.44299304679851 -1.94299304679851 \\
2.44766656544747 -1.94766656544747 \\
2.4934603986634 -1.9934603986634 \\
2.49813391731237 -1.99813391731237 \\
2.5439277505283 -2.0439277505283 \\
2.54860126917726 -2.04860126917726 \\
2.5943951023932 -2.09439510239319 \\
2.59906862104216 -2.09906862104216 \\
2.64486245425809 -2.14486245425809 \\
2.64953597290705 -2.14953597290706 \\
2.69532980612299 -2.19532980612299 \\
2.70000332477195 -2.20000332477195 \\
2.74579715798788 -2.24579715798788 \\
2.75047067663685 -2.25047067663685 \\
2.79626450985278 -2.29626450985278 \\
2.80093802850174 -2.30093802850174 \\
2.84673186171768 -2.34673186171768 \\
2.85140538036664 -2.35140538036664 \\
2.89719921358257 -2.39719921358257 \\
2.90187273223154 -2.40187273223154 \\
2.94766656544747 -2.44766656544747 \\
2.95234008409643 -2.45234008409643 \\
2.99813391731237 -2.49813391731237 \\
3.00280743596133 -2.50280743596133 \\
3.04860126917726 -2.54860126917726 \\
3.05327478782622 -2.55327478782622 \\
3.09906862104216 -2.59906862104216 \\
3.10374213969112 -2.60374213969112 \\
3.14953597290705 -2.64953597290705 \\
3.15420949155602 -2.65420949155602 \\
3.20000332477195 -2.70000332477195 \\
3.20467684342091 -2.70467684342091 \\
3.25047067663685 -2.75047067663685 \\
3.25514419528581 -2.75514419528581 \\
3.30093802850174 -2.80093802850174 \\
3.30561154715071 -2.80561154715071 \\
3.35140538036664 -2.85140538036664 \\
3.3560788990156 -2.8560788990156 \\
3.40187273223154 -2.90187273223154 \\
3.4065462508805 -2.9065462508805 \\
3.45234008409643 -2.95234008409643 \\
3.45701360274539 -2.95701360274539 \\
3.50280743596133 -3.00280743596133 \\
3.50748095461029 -3.00748095461029 \\
3.55327478782622 -3.05327478782622 \\
3.55794830647519 -3.05794830647519 \\
3.60374213969112 -3.10374213969112 \\
3.60841565834008 -3.10841565834008 \\
3.65420949155602 -3.15420949155602 \\
3.65888301020498 -3.15888301020498 \\
3.70467684342091 -3.20467684342091 \\
3.70935036206988 -3.20935036206988 \\
3.75514419528581 -3.25514419528581 \\
3.75981771393477 -3.25981771393477 \\
3.80561154715071 -3.30561154715071 \\
3.81028506579967 -3.31028506579967 \\
3.8560788990156 -3.3560788990156 \\
3.86075241766457 -3.36075241766457 \\
3.9065462508805 -3.4065462508805 \\
3.91121976952946 -3.41121976952946 \\
3.9570136027454 -3.45701360274539 \\
3.96168712139436 -3.46168712139436 \\
4.00748095461029 -3.50748095461029 \\
4.01215447325925 -3.51215447325925 \\
4.05794830647519 -3.55794830647519 \\
4.06262182512415 -3.56262182512415 \\
4.10841565834008 -3.60841565834008 \\
4.11308917698905 -3.61308917698905 \\
4.15888301020498 -3.65888301020498 \\
4.16355652885394 -3.66355652885394 \\
4.20935036206988 -3.70935036206988 \\
4.21402388071884 -3.71402388071884 \\
4.25981771393477 -3.75981771393477 \\
4.26449123258374 -3.76449123258374 \\
4.31028506579967 -3.81028506579967 \\
4.31495858444863 -3.81495858444863 \\
4.36075241766456 -3.86075241766456 \\
4.36542593631353 -3.86542593631353 \\
4.41121976952946 -3.91121976952946 \\
4.41589328817842 -3.91589328817842 \\
4.46168712139436 -3.96168712139436 \\
4.46636064004332 -3.96636064004332 \\
4.51215447325925 -4.01215447325925 \\
4.51682799190822 -4.01682799190822 \\
4.56262182512415 -4.06262182512415 \\
4.56729534377311 -4.06729534377311 \\
4.61308917698905 -4.11308917698905 \\
4.61776269563801 -4.11776269563801 \\
4.66355652885394 -4.16355652885394 \\
4.6682300475029 -4.1682300475029 \\
4.71402388071884 -4.21402388071884 \\
4.7186973993678 -4.2186973993678 \\
4.76449123258374 -4.26449123258374 \\
4.7691647512327 -4.2691647512327 \\
4.81495858444863 -4.31495858444863 \\
4.81963210309759 -4.31963210309759 \\
4.86542593631353 -4.36542593631353 \\
4.87009945496249 -4.37009945496249 \\
4.91589328817842 -4.41589328817842 \\
4.92056680682739 -4.42056680682739 \\
4.96636064004332 -4.46636064004332 \\
4.97103415869228 -4.47103415869228 \\
5.01682799190822 -4.51682799190822 \\
5.02150151055718 -4.52150151055718 \\
5.06729534377311 -4.56729534377311 \\
5.07196886242208 -4.57196886242208 \\
5.11776269563801 -4.61776269563801 \\
5.12243621428697 -4.62243621428697 \\
5.16823004750291 -4.66823004750291 \\
5.17290356615187 -4.67290356615187 \\
5.2186973993678 -4.7186973993678 \\
5.22337091801676 -4.72337091801676 \\
5.2691647512327 -4.7691647512327 \\
5.27383826988166 -4.77383826988166 \\
5.31963210309759 -4.81963210309759 \\
5.32430562174656 -4.82430562174656 \\
5.37009945496249 -4.87009945496249 \\
5.37477297361145 -4.87477297361145 \\
5.42056680682739 -4.92056680682739 \\
5.42524032547635 -4.92524032547635 \\
5.47103415869228 -4.97103415869228 \\
5.47570767734125 -4.97570767734125 \\
5.52150151055718 -5.02150151055718 \\
5.52617502920614 -5.02617502920614 \\
5.57196886242208 -5.07196886242208 \\
5.57664238107104 -5.07664238107104 \\
5.62243621428697 -5.12243621428697 \\
5.62710973293594 -5.12710973293594 \\
5.67290356615187 -5.17290356615187 \\
5.67757708480083 -5.17757708480083 \\
5.72337091801676 -5.22337091801676 \\
5.72804443666573 -5.22804443666573 \\
5.77383826988166 -5.27383826988166 \\
5.77851178853062 -5.27851178853062 \\
5.82430562174656 -5.32430562174656 \\
5.82897914039552 -5.32897914039552 \\
5.87477297361145 -5.37477297361145 \\
5.87944649226042 -5.37944649226042 \\
5.92524032547635 -5.42524032547635 \\
5.92991384412531 -5.42991384412531 \\
5.97570767734125 -5.47570767734125 \\
5.98038119599021 -5.48038119599021 \\
6.02617502920614 -5.52617502920614 \\
6.03084854785511 -5.53084854785511 \\
6.07664238107104 -5.57664238107104 \\
6.08131589972 -5.58131589972 \\
6.12710973293594 -5.62710973293594 \\
6.1317832515849 -5.6317832515849 \\
6.17757708480083 -5.67757708480083 \\
6.18225060344979 -5.68225060344979 \\
6.22804443666573 -5.72804443666573 \\
6.23271795531469 -5.73271795531469 \\
6.27851178853062 -5.77851178853062 \\
6.28318530717959 -5.78318530717959 \\
NaN NaN \\
};

\addlegendentry{decision boundary};

\addplot [
color=green,
only marks,
mark=asterisk,
mark options={solid}
]
table[row sep=crcr]{
1 0\\
0 1\\
1 1\\
};
\addlegendentry{class 1};

\addplot [
color=red,
only marks,
mark=+,
mark options={solid}
]
table[row sep=crcr]{
0 0\\
};
\addlegendentry{class -1};

\end{axis}
\end{tikzpicture}%
	\caption{}
	\label{fig:or}
\end{figure}
\begin{figure}
	\centering
	\setlength\figureheight{6cm} 
	\setlength\figurewidth{8cm}
	% This file was created by matlab2tikz v0.4.4 running on MATLAB 8.0.
% Copyright (c) 2008--2013, Nico Schlömer <nico.schloemer@gmail.com>
% All rights reserved.
% 
% The latest updates can be retrieved from
%   http://www.mathworks.com/matlabcentral/fileexchange/22022-matlab2tikz
% where you can also make suggestions and rate matlab2tikz.
% 
\begin{tikzpicture}

\begin{axis}[%
width=\figurewidth,
height=\figureheight,
scale only axis,
xmin=-0.1,
xmax=1.1,
xlabel={${\text{x}}$},
ymin=-0.1,
ymax=1.1,
ylabel={${\text{y}}$},
title={XOR-Problem},
legend style={at={(0.5,-0.03)},anchor=north,draw=black,fill=white,legend cell align=left}
]
\addplot [
color=green,
only marks,
mark=asterisk,
mark options={solid}
]
table[row sep=crcr]{
1 0\\
0 1\\
};
\addlegendentry{decision boundary};

\addplot [
color=red,
only marks,
mark=+,
mark options={solid}
]
table[row sep=crcr]{
0 0\\
1 1\\
};
\addlegendentry{class 1};

\end{axis}
\end{tikzpicture}%
	\caption{}
	\label{fig:xor}
\end{figure}


\subsubsection{Results \& Convergence}

The algorithm with the input from the first data set needs only 9 epochs to find a linear discriminant function that has a classification rate of $100\%$ (see Figure~\ref{fig:decisionBoundary1}). But the second data set has some values that could get compared with the XOR-Problem, because they are in the dynamic range of the other class. (see Figure~\ref{fig:decisionBoundary2})


To demonstrate the convergence of the perceptron it is only useful to look at the data set with the XOR-Problem. Figure~\ref{fig:epochs} shows the development of the classification rate to the epochs. This does not converge (even with the maximum of epoches set to 1000 and more). 

\begin{figure}
	\centering
	\setlength\figureheight{6cm} 
	\setlength\figurewidth{8cm}
	% This file was created by matlab2tikz v0.4.4 running on MATLAB 8.0.
% Copyright (c) 2008--2013, Nico Schlömer <nico.schloemer@gmail.com>
% All rights reserved.
% 
% The latest updates can be retrieved from
%   http://www.mathworks.com/matlabcentral/fileexchange/22022-matlab2tikz
% where you can also make suggestions and rate matlab2tikz.
% 
\begin{tikzpicture}

\begin{axis}[%
width=\figurewidth,
height=\figureheight,
scale only axis,
xmin=0,
xmax=100,
xlabel={epochs},
ymin=0,
ymax=50,
ylabel={\%}
]
\addplot [
color=blue,
solid,
forget plot
]
table[row sep=crcr]{
1 15.5\\
2 18\\
3 28.5\\
4 28\\
5 30.5\\
6 32\\
7 26.5\\
8 32\\
9 26.5\\
10 27.5\\
11 33\\
12 30.5\\
13 14\\
14 10\\
15 13.5\\
16 10\\
17 13.5\\
18 10\\
19 10\\
20 26\\
21 10\\
22 16.5\\
23 14\\
24 10.5\\
25 13.5\\
26 12\\
27 28.5\\
28 32\\
29 14.5\\
30 12.5\\
31 11.5\\
32 28.5\\
33 15\\
34 13.5\\
35 30.5\\
36 14\\
37 10\\
38 10\\
39 13.5\\
40 10\\
41 10\\
42 26.5\\
43 10\\
44 16.5\\
45 14\\
46 10.5\\
47 13.5\\
48 12\\
49 28.5\\
50 32\\
51 14.5\\
52 12.5\\
53 11.5\\
54 28.5\\
55 15\\
56 13.5\\
57 30.5\\
58 14\\
59 10\\
60 10\\
61 13.5\\
62 10\\
63 10\\
64 26.5\\
65 10\\
66 16.5\\
67 14\\
68 10.5\\
69 13.5\\
70 12\\
71 28.5\\
72 32\\
73 14.5\\
74 12.5\\
75 11.5\\
76 28.5\\
77 15\\
78 13.5\\
79 30.5\\
80 14\\
81 10\\
82 10\\
83 13.5\\
84 10\\
85 10\\
86 26.5\\
87 10\\
88 16.5\\
89 14\\
90 10.5\\
91 13.5\\
92 12\\
93 28.5\\
94 32\\
95 14.5\\
96 12.5\\
97 11.5\\
98 28.5\\
99 15\\
100 13.5\\
};
\end{axis}
\end{tikzpicture}%
	\caption{the development of the classification rate to the epochs of the data set 2}
	\label{fig:epochs}
\end{figure}


\end{document}

% vim:foldmethod=marker
